\documentclass{article}
\usepackage{fancyhdr}
\usepackage{multicol}
\usepackage[italian]{babel}
\usepackage{makeidx}
\usepackage[a4paper, total={6in, 8in}]{geometry}
%\usepackage[%
    %a5paper,
%    papersize={5.5in,8.5in},
%    margin=0.75in,
%    top=0.75in,
%    bottom=0.75in,
    %twoside
%    ]{geometry}

\usepackage{xcolor}
\usepackage{booktabs}
\usepackage{graphicx}
\usepackage[clock]{ifsym}
\usepackage{cookingsymbols}
\usepackage{hyperref}

\raggedcolumns
\setlength{\multicolsep}{0pt}
\setlength{\columnseprule}{1pt}

\makeindex
\makeatletter

\newif\if@mainmatter \@mainmattertrue

%% Borrowed from book.cls
\newcommand\frontmatter{%
    \cleardoublepage
  \@mainmatterfalse
  \pagenumbering{roman}}
\newcommand\mainmatter{%
    \cleardoublepage
  \@mainmattertrue
  \pagenumbering{arabic}}
\makeatother

%% Vary the colors at will

\definecolor{vegcolor}{rgb}{0,0.5,0.2}
\definecolor{frzcolor}{rgb}{0,0,1}

% Your "recipes.sty" package starts here:
%% Thanks to alephzero for the excellent start:

\newcommand{\recipe}[2][]{%
    \newpage
    \lhead{}%
    \chead{}%
    \rhead{}%
    \lfoot{}%
    \rfoot{}%
    \section{#2}%
    \if###1##%
    \else
        \begin{center}
            \parbox{0.75\linewidth}{\raggedright\itshape#1}%
        \end{center}
    \fi
}
\newcommand{\serves}[2][Porzioni \Fork\Dish\Knife]{%
    \chead{#1 #2}}
\newcommand{\vegetarian}{%
    \rhead{\large\color{vegcolor}\textbf{V}}}
\newcommand{\cbt}{%
    \rhead{\large\color{vegcolor}\textbf{CBT}}}
\newcommand{\freeze}{%
    \lhead{\large\color{frzcolor}\textbf{F}}}
%% Optional arguments for alternate names for these:
\newcommand{\preptime}[2][\Interval]{%
    \lfoot{#1 #2}%
}
\newcommand{\cooktime}[2][Cottura]{%
    \rfoot{#1: #2}%
}
\newcommand{\temp}[1]{%
    $#1^\circ$C}
%% Optional argument is the width of the graphic, default = 1in
\newcommand{\showit}[3][1in]{%
    \begin{center}
        \bigskip
            \includegraphics[width=#1]{#2}%
            \par
            \medskip
            \emph{#3}
            \par
        \end{center}%
    }

%% Optional argument for a  heading within the ingredients section
\newcommand{\ingredients}[1][]{%
    \if###1##%
        {\color{red}\Large\textbf{Ingredienti}}%
    \else
        \emph{#1}%
    \fi
}

%% Use \obeylines to minimize markup
\newenvironment{ingreds}{%
    \parindent0pt
    \noindent
    \ingredients
    \par
    \smallskip
    \begin{multicols}{2}
    \leftskip1em
    \rightskip0pt plus 3em
    \parskip=0.25em
    \obeylines
    \everypar={\hangindent2em}
}{%
    \end{multicols}%
    \medskip
}

\newcounter{stepnum}

%% Optional argument for an italicized pre-step
%% Also use obeylines to minimize markup here as well
\newenvironment{method}[1][]{%
    \setcounter{stepnum}{0}
    \noindent
    {\color{red}\Large\textbf{Preparazione}}%
    \par
    \par
    \smallskip
    \if###1##%
    \else
        \noindent
        \emph{#1}
        \par
    \fi
    \begingroup
    \parindent0pt
    \parskip0.25em
% La riga seguente è l'indentazione del testo della ricetta
%        \leftskip2em
% La riga seguente numera i passaggi della ricetta
%    \everypar={\llap{\stepcounter{stepnum}\hbox to2em{\thestepnum.\hfill}}}
}{%
    \par
    \endgroup}


%% Optional argument for an italicized pre-step
%% Also use obeylines to minimize markup here as well
\newenvironment{note}[1][]{%
    \setcounter{stepnum}{0}
    \noindent
\newline
    {\color{red}\Large\textbf{Note}}%
    \par
    \smallskip
    \if###1##%
    \else
        \noindent
        \emph{#1}
        \par
    \fi
    \begingroup
    \parindent0pt
    \parskip0.25em
%        \leftskip2em
%    \everypar={\llap{\stepcounter{stepnum}\hbox to2em{\thestepnum.\hfill}}}
}{%
    \par
    \endgroup}


\pagestyle{fancy}
% End of "recipes.sty"



\begin{document}

\frontmatter
\tableofcontents

\mainmatter
%\part{Introduzione}
%\section{Cosa contiene questo testo}




\section{Temperature di cottura}
Conoscere la temperatura di cottura al cuore di un alimento è fondamentale per poter replicare le ricette senza possibilità di errore e per garantire sempre il miglior risultato possibile anche secondo il proprio gusto personale.

La tabella \ref{temperature-al-cuore} è stata elaborata sul mio gusto personale e facendo prove diverse.
La tabella
\begin{center}
\begin{tabular}{ll}
\toprule
Alcaloide & Origine \\
\midrule
atropina & belladonna \\
morfina
& papavero \\
nicotina & tabacco \\
\bottomrule
\end{tabular}
\end{center}
mostra l’origine di tre alcaloidi.



\begin{table}
\begin{tabular}{llp{0.6\textwidth}}
\toprule
Alimento				&	Temp(°C)		&		Note		\\
\midrule
Hamburger			& 	62°C 		& Ideale per una cottura media\\
Braciole di maiale	& 	72°C 		& Questa temperatura è adatta per la cottura del maiale in genere. Con questa temperatura il maiale risulterà sicuro e succoso.\\
Pollo ripieno 		& 	71°C 		& Temperatura ideale per qualunque volatile ripieno\\
Polpettone 			&	75°C 		& Temperatura ideale se si aggiunge pane ammollato nel latte, altrimenti meglio abbassare la temperatura\\
\bottomrule
\end{tabular}
\label{temperature-al-cuore}
\caption{Temperature di cottura al cuore dell'alimento ottimali}
\end{table}



%\newpage
%\part{Pane e lievitati}
%\recipe[]{Panini da hamburgher veloci}
\serves{8}
\preptime{2 ore e 30}
\cooktime[Cottura]{30 minuti}
\begin{ingreds}
	500g di farina 0
	250g di acqua fredda
	30g di strutto
	10g di sale
	25g di lievito
\end{ingreds}

\begin{method}
	Metti tutti gli ingredienti nella planetaria e impasta per 10 minuti.

	Dai un paio di pieghe alla pasta e forma le palline. Per mini hamburgher fai palline da 30g per hamburgher tradizionali fai palline da 120g. Questa operazione si chiama pirlatura. Si tratta di richiudere la pasta su se stessa come un sacchetto in modo che la gabbia glutinica intrappoli le bolle d'aria che si sviluppano durante la lievitazione.

	Fai lievitare fino al raddoppio del volume. Servirà circa 1 ora e 30 minuti, ma questo tempo può variare in base alla temperatura e alla quantità di lievito che decidi di utilizzare.
	
	Quando i panini sono lievitati, se vuoi, puoi spennellare la superfice con tuorlo d'uovo e acqua o latte e successivamente cospargere semi si sesamo o altri semi. Questa operazione, oltre a rendere i panini più belli, conferirà ai panini una nota profumata interessante.

	Cuoci i panini in forno ventilato a \temp{190} per 15 minuti.

\end{method}

%\recipe[Stupite i vostri commensali con questi deliziosi crackers fatti in casa.]{Crackers al sesamo}
\serves{4}
\preptime{1 ora}
\cooktime[Cottura]{30 minuti}
\begin{ingreds}
	280g di farina
	50g di acqua
	50g di olio evo
	100g di vino bianco
	50g di sesamo
	5g di sale
\end{ingreds}

\begin{method}
	Metti la farina, il sale, il sesamo nella bull della planetaria e mescola. Aggiungi la parte liquida e impasta leggermente con la foglia poi fai riposare la pasta per 20 minuti coperta con una pellicola.
	Tira la pasta con la macchinetta fino alla misura 3 senza ripiegare. Taglia i crackers e mettili sulla leccarda ricoperta di carta da forno.

	Cuoci in forno ventilato a \temp{180} per 25/30 minuti.
\end {method}



\newpage
\part{Antipasti}
\recipe[I mini hamburger di ricciola sono un piatto delicato e leggero adatto per un aperitivo in compagnia.]{Mini hamburger di ricciola}
\serves{4}
\preptime{1 ora}
\cooktime[Cottura]{5 minuti}
\begin{ingreds}
	200g di ricciola
	4 foglie di alloro
	1 pezzo di porro
	Pane da hamburgher
	
\columnbreak
\ingredients[Per la panna acida:]
	1 cipolla
	1 bicchiere di aceto
	1 bicchiere di vino
	125g di panna fresca
\end{ingreds}

\begin{method}
Per prima cosa prepara la panna acida. Fai stufare la cipolla tagliata a striscioline, poi aggiungi il vino e l'aceto e fai ridurre lentamente. Nel frattempo in un padellino fai ridurre la panna. Setaccia e spremi la cipolla. Aggiungila alla panna e fai riposare in frigorifero per qualche ora. Puoi preparare la panna acida il giorno prima conservandola in frigorifero.

Sfiletta la ricciola e prepara una tartare battuta a coltello. Con il coppapasta prepara dei mini hamburger serrandoli con mezza foglia di alloro sui due lati. Lascia riposare gli hamburger in frigorifero.

Taglia il porro a juilenne sottili, cospargilo di farina e friggilo in olio di semi di girasole a \temp{170}.

Taglia il pane con il coppapasta che hai usato per preparare gli hamburger e fallo tostare in forno. Cuoci gli hamburger in una padella antiaderente senza esagerare con la cottura. Componi i panini con la panna acida, l'hamburger di ricciola al quale avrai tolto le foglie di alloro e il porro fritto. Puoi fermare i panini con uno stuzzicadente.
\end {method}
	\begin{note}
		Puoi sistituire la ricciola con branzino o orata.
	\end{note}


\newpage
\part{Primi}
\recipe[La gallinella di mare è un pesce che è facile trovare in pescheria ad un prezzo molto basso. Il risotto preparato con questo pesce è un piatto delicato, gustoso.]{Risotto con gallinella di mare}
\serves{4}
\preptime{1 ora}
\cooktime[Cottura]{30 minuti}
\autore{Max}
\begin{ingreds}
	350g di riso carnaroli
	4 gallinelle di mare di media pezzatura
	1 bicchiere di vino bianco
	1 scalogno
	Prezzemolo
	Scorza di limone
\columnbreak
\ingredients[Per il fumetto:]
	1/2 gambo di sedano
	1/2 carota
	1/2 cipolla
	Peperoncino
	1 bicchiere di vino bianco
	Olio evo 

\end{ingreds}

\begin{method}
	Pulisci accuratamente la gallinella di mare togliendo le branchie e lavando abbondantemente sotto acqua corrente in modo da eliminare le parti più scure. Ricava per ogni pesce due filetti privati delle lische e della pelle e riponili in frigorifero. Puoi togliere le lische centrali con una pinzetta, oppure puoi  asportare con un coltello tutta la porzione con le lische utilizzando per il fumetto.

	In una casseruola fai soffriggere nell'olio extravergine d’oliva l'aglio e il peperoncino. Quando la padella è bella calda butta tutti gli scarti del pesce e fai andare a fuoco vivo per alcuni minuti. Non preoccuparti se il pesce si brucia leggermente: donerà al piatto un fantastico profumo di mare. Quando il pesce sarà completamente rosolato, fai sfumare con il vino bianco e aggiungi acqua fredda fino a coprirlo completamente. In questa fase è importante creare uno shock termico dal caldo al freddo per estrarre più sapore possibile dagli scarti; per migliorare questa operazione puoi aggiungere del ghiaccio all'acqua fredda.
	
	Abbassa la fiamma al minimo e lascia sobbollire per 15 minuti poi filtra. Questo brodo servirà per portare a cottura il risotto.

	In una casseruola fai rosolare leggermente i filetti di gallinella tagliati a tocchetti poi toglili e tienili da parte. Aggiungi alla stessa padella lo scalogno tritato, fai soffriggere leggermente, poi aggiungi il riso e fallo tostare. Fai sfumare il riso con il vino bianco e portalo a cottura con il brodo di pesce. A cottura ultimata, aggiusta di sale, aggiungi i pezzi di gallinella e fai mantecare con un po’ di olio extravergine di oliva.

	Guarisci il piatto con una spolverata di prezzemolo e scorza di limone.

\end {method}



\part{Secondi di pesce}
\recipe[Il salmoriglio è una salsa delicata, di origini siciliane, a base di olio extravergine di oliva limone e origano. In questa ricetta viene preparata a bagnomaria per estrarre al meglio gli olii essenziali profumati delle erbe aromatiche.]{Pesce spada al salmoriglio}
\serves{4}
\preptime{1 ora}
\cooktime[Cottura]{20 minuti}

\begin{ingreds}
	\ingredients[Per il salmoriglio:]
	1 limone con buccia edibile
	1/2 bicchiere di olio e.v.o.
	2 foglie di alloro
	1 spicchio d'aglio
	origano
	pepe
	sale
\columnbreak
     	600g Pesce spada
     	Olio e.v.o.
	Sale
\end{ingreds}

\begin{method}
Taglia il pesce spada a tocchetti di 3-4 centimetri, ungilo accuratamente e riponilo in frigorifero.

Inizia a preparare il salmoriglio mettendo in una bull, adatta al bagnomaria, la scorza grattigiata e il succo di un limone, un pizzico di sale, un po' di pepe e un goccio di acqua calda. Aggiungi l'aglio tritato, l'origano essiccato e le due foglie di alloro spezzettate. Mescola con una frusta per fare sciogliere bene il sale. Aggiungi l'olio extravergine di oliva e cuoci a bagnomaria per circa 15 minuti mescolando di tanto in tanto con la frusta. A cottura ultimata, togli la bull dal fuoco e filtra la salsa con un colino.

Cuoci in una padella antiaderente a fuoco alto il pesce spada, lasciandolo leggermente crudo all'interno. Occorreranno pochi minuti per lato. Lascia riposare il pesce cotto per alcuni minuti coprendolo con carta di alluminio. Taglia i pezzi di pesce in due e disponilo in un piatto poi bagna il pesce con il salmoriglio.

\end {method}




\recipe[Il tataki è una cottura particolare che prevede di scottare gli alimenti a temperatura altissima per poi immergerli in acqua e ghiaccio per bloccare la cottura.]{Tataki di tonno fagioli e cipolla}
\serves{4}
\preptime{1 ora}
\cooktime[Cottura]{5 minuti}
\autore{Max}
\begin{ingreds}
	500g tonno fresco \index{tonno}
	1 arancia
	1 rametto di rosmarino \index{rosmarino}
	1 spicchio d'aglio
	olio evo \index{olio!olio e.v.o.}
	sale
\columnbreak
\ingredients[Per la salsa:]
	120g di fagioli canellini lessati \index{fagioli}
	brodo vegetale
	limone \index{limone}
	olio evo \index{olio!olio e.v.o.}
	sale
	cipolla \index{cipolla}
\end{ingreds}

\begin{method}

Taglia il tonno a parallelepipedi con il lato piccolo di 3x3 centimetri. Metti i filetti a marinare per qualche ora in frigorifero con le scorze degli agrumi e l'olio extravergine di oliva.

Scalda una padella antiaderente o di ferro a temperatura altissima e cuoci il tonno per pochi secondi per ogni lato. Successivamente immergi il tonno in acqua e ghiaccio per fermare la cottura. Asciuga il tonno e mettilo a marinare con rosmarino e aglio coprendolo completamente di olio extravergine di oliva.

Prepara la salsa frullando i fagioli con brodo vegetale, succo di limone, sale e olio extravergine di oliva. Setaccia la salsa così ottenuta con un colino.

Friggi la cipolla leggermente infarinata.

Componi il piatto con la salsa di fagioli sul fondo, dei cubetti di tonno, la cipolla fritta, qualche granello di sale e un giro di olio evo.
\end{method}


\newpage
\part{Secondi di carne}
\recipe[]{Nuggets di pollo}
\serves{4}
\preptime{1 ora}
\cooktime{4 minuti}
\autore{Max}
\begin{ingreds}
	500g  di petto di pollo \index{pollo}
	45g di cipolla bianca \index{cipolla}
	80g di pangrattato\index{pangrattato}
	1 cucchiaino di sale fino
	1 cucchiaino di senape \index{senape}
	2 cucchiai di salsa di soia \index{salsa di soia}
	Uno spicchio d'aglio


\columnbreak
\ingredients[Per la panatura:]
	80g di farina \index{farina}
	80g di latte \index{latte}
	1 uovo \index{uova}
	un pizzico di sale fino
	pangrattato \index{pangrattato}
\end{ingreds}

\begin{method}
Passa al mixer il pollo con la cipolla e l'aglio schiacciato; in alternativa puoi usare il tritacarne. Aggiungi il sale, la senape, la salsa di soia e il pangrattato poi impasta per amalgamare tutti gli ingredienti.

Prepara le polpette schiacciandole leggermente per dare la classica forma di pepita e mettile in un vassoio. Non produrre delle pepite eccessivamente grandi poiché la panatura avrà una parte consistente della pepita. Dovresti riuscire a preparare circa 30 polpette. Per evitare che la carne si attacchi alle mani puoi usare dei guanti monouso oppure puoi inumidire le mani.

Prepara la pastella mescolando con una frusta la farina con il latte e l'uovo. Aggiungi un pizzico di sale. Tuffa i \textit{nuggets} nella pastella, poi nel pangrattato.

Friggi a \temp{175} per 3 minuti.

\end {method}

\subsection*{Note}
		Puoi congelare i \textit{nuggets} dopo averli impanati e cuocerli da congelati portando il tempo di cottura a 5 minuti.

		Se hai molti \textit{nuggets} da cuocere, puoi tenere un contenitore con il pollo già fritto in forno statico alla temperatura di \temp{60} avendo cura di inserire un cucchiaio di legno nello sportello del forno per garantire la fuoriuscita di vapore oppure aprendo lo sportello abbastanza spesso. In questo modo potrai servire tutto il fritto assieme ancora caldo.



\recipe[]{Polpettone di manzo}
\serves{4}
\preptime{1 ora}
\cooktime{45 minuti}
\begin{ingreds}
	600g di macinato di manzo + una salsiccia di maiale
	pecorino grattugiato
	parmigiano grattugiato
	pane ammollato nel latte
	2 uova
	sale
	pepe
	erbe aromatiche
	prosciutto cotto
	provola o scamorza
	pangrattato
\end{ingreds}

\begin{method}
Mescola il macinato di manzo con la mollica ammollata nel latte e strizzata, le uova, il formaggio grattugiato e le erbe aromatiche. Regola di sale e pepe.

Stendi il macinato su un foglio di carta da forno in modo da formare un rettangolo alto circa 1,5 centimetri. Stendi sopra il prosciutto cotto e la scamorza, poi arrotola aiutandoti con la carta da forno. Ricorda di sigillare bene anche le estremità.

Ungi la superficie del polpettone con olio extravergine d'oliva e spolvera leggermente con del pane grattato.

Cuoci in forno statico a \temp{200} fino al raggiungimento dei \temp{75} al cuore; impiegherà circa 45 minuti.
\end {method}



%
%\newpage
%\part{BBQ}
%\recipe[Le eliche di pollo al curry sono degli spiedini con un sapore etnico, adatti sia come antipasto che come aperitivo.]{Eliche di pollo al curry}
\serves{4}
\preptime{1 ora}
\cooktime[Cottura]{6 minuti}
\begin{ingreds}
	1 petto di pollo
	sale
	olio evo
	curry
	pan grattato
\end{ingreds}

\begin{method}
	Taglia il petto di pollo a striscie e mettile a marinare con olio extravergine d'oliva, un pizzico di sale e abbondante curry.

	Prepara degli spiedini disponendo a elica il pollo, poi cospargi leggermente con il pan grattato.

	Porta il barbecue alla massima temperatura e cuoci gli spiedini 3 minuti per lato.
\end {method}

%\recipe[Sono praticamente una droga.]{Smoked chicken wings}
\serves{4}
\preptime{1 ora}
\cooktime[Cottura]{40 minuti}
\begin{ingreds}
	12 ali di pollo
	salsa bbq
\columnbreak
\ingredients[Per la marinatura:]
	175g di burro
	1 cucchiaio di paprika
	succo di un limone
	1 cucchiaino di senape
	1 cucchiaino di sale
	1 cucchiaio di zucchero
	1 bicchiere di burbon
	1 cucchiaino di concentrato di pomodoro
\end{ingreds}

\begin{method}
	Togli la punta delle ali di pollo e dividile a metà. In un tegamino sciogli il burro, aggiungi il limone, e gli altri ingredienti per la marinatura. Attendi che l'acool contenuto nel burbon sia evaporato completamente, poi fai raffreddare.

	Metti a marinare il pollo per alcune ore in frigorifero con la salsa appena preparata.

	Stabilizza il barbecue a \temp{140} e cuoci le alette 20 minuti per lato.

	Alza la temperatura del barbecue, spennella le alette con salsa barbecue e cuoci per 5 minuti.
\end {method}

%\recipe[Un secondo di carne saporito.]{Bombette pugliesi}
\serves{4}
\preptime{1 ora}
\cooktime[Cottura]{40 minuti}
\vegetarian
\freeze
\begin{ingreds}
	12 fette sottili di coppa di maiale
	400g di pancetta
	prezzemolo
	aglio
	pepe
	sale
	caciocavallo

\end{ingreds}

\begin{method}
	In un mixer prepara un trito di prezzemolo, aglio, sale e pepe.

	Prepara le bombette stendendo, sulla fetta di coppa, una fetta di pancetta, una fetta di caciocavallo e il prezzemolo all'aglio. Arrotola l'involtino e infilzalo in uno spiedo.

	Stabilizza il barbecue alla temperatra di \temp{140} e cuoci gli spidini per circa 40 minuti.
\end {method}

%\recipe[Profumatissimi spiedini da fare al barbecue con una ricetta presa da bbq4all]{Involtini di carne alla siciliana}
\serves{4}
\preptime{1 ora}
\cooktime[Cottura]{15 minuti}
\begin{ingreds}
	6 fette di carpaccio di manzo
	una cipolla rossa
	scamorza
	prosciutto cotto 
	uvetta
	pan grattato
	qualche foglia di alloro
\end{ingreds}

\begin{method}
	Taglia il carpaccio in modo da formare delle fette triangolari. In genere per ogni fetta di carpaccio è possibile ricavare 2 triangoli.

	Prepara la farcia per gli involtini mescolando l'uvetta ammollata e leggermente tritata con il prosciutto cotto tagliato a cubettini, la scamorza e un po' di pan grattato.

	Prepara gli involtini con la farcia e mettili in uno spiedo alternati da un petalo di cipolla e una foglia di alloro.

	Ungi leggermente gli  spiedini e dai una spolverata con il pan grattato, poi cuoci al barbecue fino allo scioglimento completo della farcia.

\end {method}



%
%\newpage
%\part{Cottura a bassa temperatura (CBT)}
%
%\recipe[Una ricetta che prevede più passaggi in diversi giorni ma che garantisce grandi soddisfazioni. Indispensabili la macchina per sotto vuoto e un roner per cottura in bagno a temperatura controllata.]{Stinco di maiale (CBT)}
\serves{4}
\preptime{4 giorni}
\cooktime[Cottura]{24 ore}
\cbt
\begin{ingreds}
	\ingredients[Per la cottura dello stinco:]
	2 stinchi di maiale
	2 cucchiai di senape
	2 cucchiai di rub
	salsa bbq
	\ingredients[Per la salsa dello stinco:]
	1/2 costa di sedano
	1/2 carota
	1/2 cipolla
	1 bicchiere di vino rosso
	1 cucchiaio di farina

\columnbreak
	\ingredients[Per la salamoia]
	1l acqua
	20g sale
	10g miele
	8 cucchiai di aceto
	3 foglie di alloro
	2 bacche di ginepro
	1 stella di anice
	8 cucchiai di aceto
\end{ingreds}

\begin{method}
	Il \underline{primo giorno}. Metti a bollire per alcuni minuti gli ingredienti per la salamoia utilizzando la quantità di acqua necessaria per coprire completamente gli stinchi. Di conseguenza calcola in proporzione gli altri ingredienti. Fai raffreddare la salamoia e immergi gli stinchi per 24 ore.

	Il \underline{secondo giorno}. Scola gli stinchi dalla salamoia e asciugali con cura. Cospargi la superficie con un leggero strato di senape e successivamente il rub che dovrà contenere sale, paprika, zucchero e aromi. Metti sotto vuoto gli stinchi separatamente e cuochi in un bagno alla temperatura di \temp{68} per 24 ore.

	Il \underline{terzo giorno}. Togli gli stinchi dal bagno termostatico e immergi in acqua, ghiaccio e sale per abbassare la temperatura più rapidamente possibile. Il bagno di acqua fredda dovrebbe essere ad una temperatura uguale o inferiore a \temp{3}. Lascia in bagno per un ora o più poi metti in frigo per almento 1 giorno.

	Il \underline{quarto giorno} (o quando decidi di mangiare gli stinchi). Riscalda il forno a \temp{250} in modalità ventilata. Togli gli stinchi dalle buste sottovuoto e recupera tutti i liquidi. Asciuga gli stinchi e cospargili di olio e infornali fino a che non raggiungono la temperatura al cuore di \temp{45}. Occorreranno circa 35 minuti. Se vuoi gli untimi 5 minuti puoi cospargere gli stinchi di salsa bbq e riprendere la cottura.

In una padella antiaderente fai un soffritto di sedano, carota e cipolla. Aggiuggi una spolverata di farina, fai sfumare con il vino rosso, poi aggiungi i liquidi di cottura degli stinchi. Quando hai raggiunto la consistenza desiderata frulla la salsa.

Metti gli stinchi nel piatto e cospargi con la salsa al vino rosso.
\end {method}




%\recipe[Una versione paticolare del classico vitello tonnato.]{Vitello tonnato (CBT)}
\serves{4}
\preptime{1 ora}
\cooktime[Cottura]{30 minuti}
\cbt
\begin{ingreds}
	500g di girello di vitello
	erbe aromatiche
	maionese
	pan grattato
	peperoncino
	capperi sott'aceto
	acciughe
	cetriolini sott'aceto
	olio evo
\end{ingreds}

\begin{method}
	Massaggia il girello di vitello con olio extravergine d'oliva e cospargi con le erbe aromatiche e il sale. Mettiolo sotto vuoto e cuoci in un bagno termostatato a \temp{58} per 4-5 ore.

	In un padellino scalda un po di olio extravergine d'oliva con il peperoncino, aggiungi il pan grattato e tosta leggermente.

	Affetta il girello molto sottile, dipsoni le fette in un piatto eguarnisci con alcuni fiocchetti di maionese, un po di pan grattato, qualche pezzetto di acciugna, capperi e cetriolini.
\end {method}

%
\newpage
\part{Dolci}
%\recipe[I pancake sono la ricetta tipica della colazione o merenda americana.]{Pancake}
\serves{4}
\preptime{20 minuti}
\cooktime[Cottura]{10 minuti}
\begin{ingreds}
	200g farina
	187g latte tiepido
	30g aceto
	25g zucchero
	10g lievito per dolci
	5g bicarbonato
	2 uova
	un pizzico di sale
\end{ingreds}

\begin{method}
	Mescola il latte con il desto degli ingredienti liquidi, poi settaccia la farina e il lievito nella bull con i liquidi. Mescola energicamente con la frusta.

	Cuoci in padella antiaderente senza aggiungere grassi a fuoco medio.

	Fai asciugare i pancake su una griglia. Puoi farcire i pancake con frutta fresca e zucchero a velo, oppure nutella o marmellata, oppure il grande classico dello sciroppo d'acero.
\end {method}




\recipe[Questa è una ricetta di famiglia che realizziamo una volta all'anno nel periodo delle feste Natalizie. Prima delle festività ne prepariamo svariati chili da regalare ad amici e parenti. Si conservano per almeno un mese in ambiente fresco e asciutto.]{Biscotti di Natale}
\serves{4}
\preptime{2 ora}
\cooktime{25 minuti}
\autore{Max}
\begin{ingreds}
	500g farina \index{farina} tipo 1
	230g noci\index{noci} e \index{nocciole}
	200g cioccolata\index{cioccolata}
	200g burro\index{burro} (puoi abbassare 180g)
	200g zucchero (puoi abbassare a 180g)
	3 uova \index{uova}
	1 bustina di lievito
	un pizzico di sale
	aromi (vanillina, vaniglia, limone)

\end{ingreds}

\begin{method}
Puoi realizzare l'impasto per questi biscotti anche con la planetaria utilizzando lo strumento a forma di foglia.

Mescola le uova con lo zucchero, poi aggiungi il burro a temperatura ambiente e continua a mescolare. Aggiungi qualche goccia di olio essenziale di limone o vaniglia. Aggiungi la farina, il lievito.

Quando l'impasto risulta omogeneo aggiungi le noci, le nocciole e, per ultimo, il cioccolato: distribuisci questi ultimi ingredienti senza impastare troppo per non sciogliere il cioccolato.
	
Prepara dei salsicciotti di circa 4cm di diametro aiutandoti con un pezzo di carta da forno che utilizzerai per avvolgere il salsicciotto stesso e riporlo successivamente nel congelatore. I biscotti dovranno rimanere nel congelatore per un minimo di 30 minuti, oppure puoi congelarli completamente e cuocerli in un secondo tempo.
	
Taglia biscotti e cuoci a \temp{185} per 20/25 minuti.

Fai raffreddare.

\end{method}

\subsection*{Note}
		Puoi sostituire lo zucchero con 20g di stevia ma dovrai aggiungere circa mezzo bicchiere di latte e cuocere per circa 10 minuti in più.
		
		Con 500g di farina puoi produrre 4 salsicciotti di biscotti.




%\newpage
%\part{Indici}
%\printindex
\end{document}
