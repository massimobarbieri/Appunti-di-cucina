\section{Temperature di cottura}
Conoscere la temperatura di cottura al cuore di un alimento ti permette di ottenere sempre, con certezza matematica, una cottura perfetta. Poiché intervengono altri parametri a costruire il gusto (marinatura, rosolatura, aromi), questo non vuole dire che la tua ricetta sarà perfetta, ma almeno avrai un dato in più per sapere se un pezzo di carne è rimasto in forno per un tempo sufficiente o se un panino è cotto.

Questa tabella è stata elaborata sul mio gusto personale e facendo prove diverse.





\begin{tabular}{|l|l|l|}
\hline
Hamburger & 62°C & Per una cottura media, puoi abbassare se lo vuoi più al sangue\\
\hline
Braciole di maiale & 72°C & Questa temperatura è adatta per la cottura del maiale in genere\\
\hline
	Pollo ripieno & 71°C & Temperatura ideale per qualunque volatile ripieno\\
	\hline
	Polpettone & 75°C & Se si aggiunge pane ammollato nel latte, se no meno\\
	\hline
\end{tabular}
