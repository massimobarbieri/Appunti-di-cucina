\recipe[Il risotto alla gallinella è un piatto delicato, gustoso ed economico.]{Risotto alla gallinella}
\serves{4}
\preptime{1 ora}
\cooktime[Cottura]{30 minuti}
\begin{ingreds}
	350g di riso carnaroli
	4 gallinelle medie
	1 bicchiere di vino bianco
	1 scalogno
	Prezzemolo
	Scorza di limone
\columnbreak
\ingredients[Per il fumetto:]
	1/2 gambo di sedano
	1/2 carota
	1/2 cipolla
	Peperoncino
	1 bicchiere di vino bianco
	Olio evo 

\end{ingreds}

\begin{method}
	Pulisci accuratamente la gallinella togliendo le branchie lavando abbondantemente sotto acqua corrente in modo da eliminare le parti più scure. Ricava per ogni pesce due filetti privati delle lische e della pelle e riponili in frigorifero.

	In una casseruola fai soffriggere l'olio extravergine d’oliva, l'aglio e il peperoncino. Quando la padella è bella calda butta tutti gli scarti del pesce e fai andare a fuoco vivo per alcuni minuti. Non preoccuparti se il pesce si brucia leggermente: donerà al piatto un fantastico profumo di mare. Quando il pesce sarà completamente rosolato, fai sfumare con il vino bianco e aggiungi acqua fredda fino a coprirlo completamente. Riporta lentamente in temperatura poi lascia sobbollire per 15 minuti e filtra. Questo brodo servirà per portare a cottura il risotto.

	In una casseruola fai rosolare leggermente i filetti di gallinella tagliati a tocchetti poi toglili e tienili da parte. Aggiungi alla stessa padella lo scalogno tritato, fai soffriggere leggermente, poi aggiungi il riso e fallo tostare. Fai sfumare il riso con il vino bianco e portalo a cottura con il brodo di pesce. A cottura ultimata aggiungi i pezzi di gallinella e fai mantecare con un po’ di olio extravergine di oliva.

	Guarisci il piatto con una spolverata di prezzemolo e scorza di limone.

\end {method}

