\recipe[I mini hamburger di orata sono un piatto delicato e leggero adatto per un aperitivo in compagnia.]{Mini hamburger di orata}
\serves{4}
\preptime{1 ora}
\cooktime{5 minuti}
\autore{Max}
\begin{ingreds}
	200g orata
	4 foglie di alloro
	1 pezzo di porro
	Pane da hamburgher (Vedi ricetta \ref{panini-hamburger})
	
\columnbreak
\ingredients[Per la panna acida:]
	1 cipolla
	1 bicchiere di aceto
	1 bicchiere di vino
	125g di panna fresca
\end{ingreds}

\begin{method}
Per prima cosa prepara la panna acida. Fai stufare la cipolla tagliata a striscioline, poi aggiungi il vino e l'aceto e fai ridurre lentamente. Nel frattempo in un padellino fai ridurre la panna. Setaccia e spremi la cipolla. Aggiungila alla panna e fai riposare in frigorifero per qualche ora. Puoi preparare la panna acida il giorno prima conservandola in frigorifero.

Sfiletta l'orata e prepara una tartare battuta a coltello. Con il coppapasta prepara dei mini hamburger serrandoli con mezza foglia di alloro sui due lati. Lascia riposare gli hamburger in frigorifero.

Taglia il porro a juilenne sottili, cospargilo di farina e friggilo in olio di semi di girasole a \temp{170}.

Taglia il pane con il coppapasta che hai usato per preparare gli hamburger e fallo tostare in forno. Cuoci gli hamburger in una padella antiaderente senza esagerare con la cottura. Componi i panini con la panna acida, l'hamburger di orata al quale avrai tolto le foglie di alloro e il porro fritto. Puoi fermare i panini con uno stuzzicadente.
\end {method}
	\begin{note}
		Puoi sistituire la ricciola con branzino o orata.
	\end{note}
