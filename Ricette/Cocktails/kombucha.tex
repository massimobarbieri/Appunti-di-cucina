\recipe[Considerato un elisir di giovinezza, il Kombucha è un fermentato di tea zuccherato ricco di probiotici]{Kombucha}
\serves{4}%<----Numero di porzioni
\preptime{10 minuti}%<---Tempo di preparazione
\cooktime[]{5 minuti}%<-----Tempo di cottura
%\autore{QUI-IL_NOME}
\begin{ingreds}
	% Qui gli ingredienti uno sotto all'altro
10g di tè verde
70/80g di zucchero
1 l di acqua
1 scoby
100g di kombucha attivo

% Se vuoi aggiungere gli ingredienti per una preparazione
% specifica della ricetta usa questi 2 comandi
%\columnbreak
%\ingredients[Per il ripieno:]

\end{ingreds}

\begin{method}


% Esempio di tabella ---------->
%\begin{table}[h]
%\begin{tabular}{lcc}
%\toprule
%	Ingredienti	&	Peso(g)	&	Peso(g)\\
%\midrule
%	Paprika		&	40	&	60	\\
%	Sale		&	33	&	50	\\
%	Zucchero	&	20	&	30	\\
%	Cumino		&	2	&	3	\\
%	Pepe		&	6	&	9	\\
%	Aglio		&	12	&	18	\\
%	Origano		&	2	&	3	\\
%\bottomrule
%\end{tabular}
%\end{table}

Per preparare il kombucha, oltre agli ingredienti indicati in precedenza, avrai bisogno di un vaso capiente in cui la bevanda possa fermentare e di una bottiglia a chiusura ermetica per la fase successiva di gasatura.

Inizia preparando un tè neutro, privo di aromi, lasciando le foglie o le bustine in infusione per circa cinque o sette minuti. Una volta pronto, aggiungi lo zucchero e mescola finché non si scioglie completamente. È importante attendere che il tè si raffreddi del tutto, poiché una temperatura troppo alta potrebbe danneggiare i lieviti e i batteri responsabili della fermentazione del kombucha.

Quando il tè è freddo, versalo nel vaso, aggiungi una parte di kombucha già pronto (circa il 10-15\% necessario per alzare l'acidità della bevanda ed impedire la formazione di muffe) e inserisci lo scoby. Copri il vaso con una garza o un panno pulito e lascialo fermentare per un periodo compreso tra dieci e quindici giorni, a seconda del livello di dolcezza o acidità che desideri ottenere. Dopo alcuni giorni, noterai la formazione di un nuovo scoby sulla superficie del liquido: al termine della fermentazione avrà raggiunto uno spessore di circa mezzo centimetro. Intorno al decimo giorno puoi iniziare ad assaggiare la bevanda per valutarne il gusto e decidere se proseguire o fermare la fermentazione.

Una volta raggiunto il sapore desiderato, rimuovi con delicatezza gli scoby e mescola bene il kombucha per ridistribuire uniformemente i lieviti che si sono depositati. Filtra la bevanda con un colino mentre la imbottigli e, se vuoi, aggiungi aromi naturali come succo d’arancia o di limone, oppure un po’ di zenzero fresco. Lascia le bottiglie a temperatura ambiente per uno o tre giorni, il tempo necessario a sviluppare la gasatura, quindi riponile in frigorifero.

Gli scoby che hai ottenuto possono essere conservati in uno “scoby hotel”, immersi nel kombucha e tenuti coperti per mantenerli vitali.

\end{method}






%\subsection*{Note}

% Figura
%\showit[1.25in]{example-image-b}{This is a picture}


