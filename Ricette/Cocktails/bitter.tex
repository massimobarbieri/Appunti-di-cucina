\recipe[]{Bitter: appunti}
\serves{-}%<----Numero di porzioni
\preptime{15 gg}%<---Tempo di preparazione
\cooktime[]{5 minuti}%<-----Tempo di cottura
\autore{Max}
\begin{ingreds}
	% Qui gli ingredienti uno sotto all'altro



% Se vuoi aggiungere gli ingredienti per una preparazione
% specifica della ricetta usa questi 2 comandi
%\columnbreak
%\ingredients[Per il ripieno:]

\end{ingreds}

\begin{method}
\underline{Bitter \#1}. Il 9/12/2022 ho messo in infusione: 300g di Vodka Stolichnaya, 4g di radice di genziana, 10 bacche di cardamomo, 1h di cumino. Fra 10 giorni filtro. Con il filtrato preparo un infuso per 5 minuti, poi il caramello. Lascio in infusione per 10 giorni. Per raggiungere il 25\% VOL dovrò aggiungere 150g di acqua, per quanto riguarda il caramello assaggerò.

Il 15/12/2022 ho fatto un infuso con 150g di acqua e il filtrato dall'alcool. Ho fatto 65g di caramello che ho sciolto nell'infuso filtrato. RISULTATO: Il sapore è perfetto, molto simile al Campari per amaro. Leggermente torbido a causa del cumino in polvere, meglio usare quello in semi. Per il resto è perfetto, eventualmente si può aggiungere una nota aromatica più spinta lasciando la stessa proporzione di angelica.
% Esempio di tabella ---------->
%\begin{table}[h]
%\begin{tabular}{lcc}
%\toprule
%	Ingredienti	&	Peso(g)	&	Peso(g)\\
%\midrule
%	Paprika		&	40	&	60	\\
%	Sale		&	33	&	50	\\
%	Zucchero	&	20	&	30	\\
%	Cumino		&	2	&	3	\\
%	Pepe		&	6	&	9	\\
%	Aglio		&	12	&	18	\\
%	Origano		&	2	&	3	\\
%\bottomrule
%\end{tabular}
%\end{table}

\end{method}

%\subsection*{Note}

% Figura
%\showit[1.25in]{example-image-b}{This is a picture}


