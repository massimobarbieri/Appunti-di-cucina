\recipe[Appunti per produrre il vermouth: ricetta da codificare]{Vermouth: appunti}
\serves{-}%<----Numero di porzioni
\preptime{1 ora}%<---Tempo di preparazione
\cooktime[]{-}%<-----Tempo di cottura
\autore{Max}
\begin{ingreds}
	% Qui gli ingredienti uno sotto all'altro
	1 litro di Vino
	10 rametti di Artemisia
	150g di Distillato
	150g Zucchero


% Se vuoi aggiungere gli ingredienti per una preparazione
% specifica della ricetta usa questi 2 comandi
%\columnbreak
%\ingredients[Per il ripieno:]

\end{ingreds}


\begin{method}
\underline{Vermounth \#1}. Il 9/12/2022 ho messo in infusione: 750ml di Valpollicella Classico Brunelli vendemmia 2021 (13\% vol), 15g di artemisia vulgaris, 2g cannella, 1g di chiodi di garofano, 0,5g di noce moscata tritata al mortaio, 2g di pepe di sichuan (di cui 1g tritato al mortaio), 6 bacche di cardamomo. Dopo 24 ore ho aggiunto 112g di Vodka Stolichnaya e 112g di zucchero semolato. Ora lascio in infusione per 10 giorni.

Nelle ricette che ho trovato su 1 litro di vino aggiungono 150g di zucchero e 150g di grappa.

% Esempio di tabella ---------->
%\begin{table}[h]
%\begin{tabular}{lcc}
%\toprule
%	Ingredienti	&	Peso(g)	&	Peso(g)\\
%\midrule
%	Paprika		&	40	&	60	\\
%	Sale		&	33	&	50	\\
%	Zucchero	&	20	&	30	\\
%	Cumino		&	2	&	3	\\
%	Pepe		&	6	&	9	\\
%	Aglio		&	12	&	18	\\
%	Origano		&	2	&	3	\\
%\bottomrule
%\end{tabular}
%\end{table}

\end{method}

%\subsection*{Note}

% Figura
%\showit[1.25in]{example-image-b}{This is a picture}


