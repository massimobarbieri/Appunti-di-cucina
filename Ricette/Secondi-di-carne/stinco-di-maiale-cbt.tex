\recipe[Una ricetta che prevede più passaggi in diversi giorni ma che garantisce grandi soddisfazioni. Indispensabili la macchina per sotto vuoto e un roner per cottura in bagno a temperatura controllata.]{Stinco di maiale (CBT)}
\serves{4}
\preptime{4 giorni}
\cooktime[Cottura]{24 ore}
\cbt
\begin{ingreds}
	\ingredients[Per la cottura dello stinco:]
	2 stinchi di maiale
	2 cucchiai di senape
	2 cucchiai di rub
	salsa bbq
	\ingredients[Per la salsa dello stinco:]
	1/2 costa di sedano
	1/2 carota
	1/2 cipolla
	1 bicchiere di vino rosso
	1 cucchiaio di farina

\columnbreak
	\ingredients[Per la salamoia]
	1l acqua
	20g sale
	10g miele
	8 cucchiai di aceto
	3 foglie di alloro
	2 bacche di ginepro
	1 stella di anice
	8 cucchiai di aceto
\end{ingreds}

\begin{method}
	Il \underline{primo giorno}. Metti a bollire per alcuni minuti gli ingredienti per la salamoia utilizzando la quantità di acqua necessaria per coprire completamente gli stinchi. Di conseguenza calcola in proporzione gli altri ingredienti. Fai raffreddare la salamoia e immergi gli stinchi per 24 ore.

	Il \underline{secondo giorno}. Scola gli stinchi dalla salamoia e asciugali con cura. Cospargi la superficie con un leggero strato di senape e successivamente il rub che dovrà contenere sale, paprika, zucchero e aromi. Metti sotto vuoto gli stinchi separatamente e cuochi in un bagno alla temperatura di \temp{68} per 24 ore.

	Il \underline{terzo giorno}. Togli gli stinchi dal bagno termostatico e immergi in acqua, ghiaccio e sale per abbassare la temperatura più rapidamente possibile. Il bagno di acqua fredda dovrebbe essere ad una temperatura uguale o inferiore a \temp{3}. Lascia in bagno per un ora o più poi metti in frigo per almento 1 giorno.

	Il \underline{quarto giorno} (o quando decidi di mangiare gli stinchi). Riscalda il forno a \temp{250} in modalità ventilata. Togli gli stinchi dalle buste sottovuoto e recupera tutti i liquidi. Asciuga gli stinchi e cospargili di olio e infornali fino a che non raggiungono la temperatura al cuore di \temp{45}. Occorreranno circa 35 minuti. Se vuoi gli untimi 5 minuti puoi cospargere gli stinchi di salsa bbq e riprendere la cottura.

In una padella antiaderente fai un soffritto di sedano, carota e cipolla. Aggiuggi una spolverata di farina, fai sfumare con il vino rosso, poi aggiungi i liquidi di cottura degli stinchi. Quando hai raggiunto la consistenza desiderata frulla la salsa.

Metti gli stinchi nel piatto e cospargi con la salsa al vino rosso.
\end {method}



