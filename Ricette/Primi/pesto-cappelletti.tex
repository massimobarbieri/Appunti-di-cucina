\recipe[Ci sono molte varianti del ripieno dei cappelletti, si può dire che ogni famiglia Reggiana abbia la propria versione. Questa è la versione di mia mamma.]{Pesto dei cappelletti}
\serves{10}%<----Numero di porzioni
\preptime{3 ore}%<---Tempo di preparazione
\cooktime[]{50 minuti}%<-----Tempo di cottura
\autore{Max (Nonna Giuli)}
\begin{ingreds}
	250g macinato misto \index{macinato}
	120g gambetto di prosciutto \index{prosciutto}
	120g mortadella \index{mortadella}
	sedano \index{sedano}
	cipolla \index{cipolla}
	burro
	2 uova \index{uova}
	Parmigiano Reggiano grattugiato \index{Parmigiano Reggiano}
	Pangrattato (se necessario)

\end{ingreds}

\begin{method}
In una casseruola scalda un po' di burro e fai soffriggere sedano e cipolla tritati finemente. Aggiungi il macinato poi fai stufare aggiungendo un po' di acqua o di brodo. Fai raffreddare e aggiungi il gambetto di prosciutto e la mortadella tritati. Aggiungi il Parmigiano Reggiano e assaggia per raggiungere il giusto grado di sapidità. Se il pesto è troppo umido puoi spolverare con poco pangrattato.

% Esempio di tabella ---------->
%\begin{table}[h]
%\begin{tabular}{lcc}
%\toprule
%	Ingredienti	&	Peso(g)	&	Peso(g)\\
%\midrule
%	Paprika		&	40	&	60	\\
%	Sale		&	33	&	50	\\
%	Zucchero	&	20	&	30	\\
%	Cumino		&	2	&	3	\\
%	Pepe		&	6	&	9	\\
%	Aglio		&	12	&	18	\\
%	Origano		&	2	&	3	\\
%\bottomrule
%\end{tabular}
%\end{table}

\end{method}

\subsection*{Note}
Le dosi per il pesto sono adatte per 7/8 uova di sfoglia.

% Figura
%\showit[1.25in]{example-image-b}{This is a picture}


