\recipe[La gallinella di mare è un pesce che è facile trovare in pescheria ad un prezzo molto basso. Il risotto preparato con questo pesce è un piatto delicato, gustoso.]{Risotto con gallinella di mare}
\serves{4}
\preptime{1 ora}
\cooktime[Cottura]{30 minuti}
\autore{Max}
\begin{ingreds}
	350g di riso carnaroli
	4 gallinelle di mare di media pezzatura
	1 bicchiere di vino bianco
	1 scalogno
	Prezzemolo
	Scorza di limone
\columnbreak
\ingredients[Per il fumetto:]
	1/2 gambo di sedano
	1/2 carota
	1/2 cipolla
	Peperoncino
	1 bicchiere di vino bianco
	Olio evo 

\end{ingreds}

\begin{method}
	Pulisci accuratamente la gallinella di mare togliendo le branchie e lavando abbondantemente sotto acqua corrente in modo da eliminare le parti più scure. Ricava per ogni pesce due filetti privati delle lische e della pelle e riponili in frigorifero. Puoi togliere le lische centrali con una pinzetta, oppure puoi  asportare con un coltello tutta la porzione con le lische utilizzando per il fumetto.

	In una casseruola fai soffriggere nell'olio extravergine d’oliva l'aglio e il peperoncino. Quando la padella è bella calda butta tutti gli scarti del pesce e fai andare a fuoco vivo per alcuni minuti. Non preoccuparti se il pesce si brucia leggermente: donerà al piatto un fantastico profumo di mare. Quando il pesce sarà completamente rosolato, fai sfumare con il vino bianco e aggiungi acqua fredda fino a coprirlo completamente. In questa fase è importante creare uno shock termico dal caldo al freddo per estrarre più sapore possibile dagli scarti; per migliorare questa operazione puoi aggiungere del ghiaccio all'acqua fredda.
	
	Abbassa la fiamma al minimo e lascia sobbollire per 15 minuti poi filtra. Questo brodo servirà per portare a cottura il risotto.

	In una casseruola fai rosolare leggermente i filetti di gallinella tagliati a tocchetti poi toglili e tienili da parte. Aggiungi alla stessa padella lo scalogno tritato, fai soffriggere leggermente, poi aggiungi il riso e fallo tostare. Fai sfumare il riso con il vino bianco e portalo a cottura con il brodo di pesce. A cottura ultimata, aggiusta di sale, aggiungi i pezzi di gallinella e fai mantecare con un po’ di olio extravergine di oliva.

	Guarisci il piatto con una spolverata di prezzemolo e scorza di limone.

\end {method}

