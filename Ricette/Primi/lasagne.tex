\recipe[Questa è la mia ricetta delle lasagne; le proporzioni degli ingredienti sono tarate sulla dimensione della mia teglia per ridurre lo spreco.]{Lasagne}
\serves{6}%<----Numero di porzioni
\preptime{3 ore}%<---Tempo di preparazione
\cooktime[]{3 ore}%<-----Tempo di cottura
\autore{Max}
%\vegetarian
%\freeze
\begin{ingreds}
\ingredients[Per la sfoglia:]
	3 \index{uova}
	300g \index{farina} 00
\ingredients[Per la besciamella:]
	750g \index{latte} intero
	75g \index{farina}
	50g burro
	sale 
	noce moscata

\columnbreak
	\index{Parmigiano Reggiano} grattugiato

\ingredients[Per il ragu:]
	500g di \index{macinato} di manzo
	1 carota
	1 gambo di sedano
	1 cipolla
	1 bicchiere di vino rosso
	Passata di pomodoro
	Aromi a piacere

\end{ingreds}

\begin{method}
Per prima cosa prepara il \index{ragù}. Fai soffriggere a fuoco vivo la carne macinata, poi aggiungi sedano, carota e cipolla tritati finemente. Quando il tutto è ben rosolato fai sfumare con un bicchiere di vino rocco corposo. Fai evaporare l'alcool e aggiungi poca passata di pomodoro. Aggiungi eventuali aromi (alloro, ginepro, chiudi di garofano). Metti il coperchio e cuoci per un paio d'ore a fuoco molto basso.

Prepara la \index{sfoglia} impastando le uova con la farina. Avvolgi in una pellicola e lascia riposare in frigorifero.

Prepara la \index{besciamella} (Vedi anche ricetta \ref{besciamella}). Sciogli il burro in una casseruola, aggiungi la farina e fai imbiondire. Scalda il latte senza raggiungere il bollore. Aggiungine poco nel tegame con la farina e mescola con una frusta. Continua ad aggiungere e mescolare per evitare la formazione di grumi. Porta a cottura la besciamella fino a raggiungere una consistenza cremosa. Regola di sale e aggiungi la noce moscata.

Tira la pasta con la macchinetta fino a raggiungere la penultima misura, ma ripassando 2 volte. Sbollenta la pasta per pochi minuti, poi stendi su un canovaccio.

Assembla le lasagne mettendo un leggero strato di besciamella sul fondo di uno stampo di ceramica o pirex. Continua con una strato di pasta, uno di besciamella, uno di ragù e una generosa spolverata di parmigiano. Continua aggiungendo almeno 5 strati.

Cuoci in forno ventilato a \temp{180} per 30 minuti o fino a che i bordi non sono leggermente bruciacchiati.

% Esempio di tabella ---------->
%\begin{table}[h]
%\begin{tabular}{lcc}
%\toprule
%	Ingredienti	&	Peso(g)	&	Peso(g)\\
%\midrule
%	Paprika		&	40	&	60	\\
%	Sale		&	33	&	50	\\
%	Zucchero	&	20	&	30	\\
%	Cumino		&	2	&	3	\\
%	Pepe		&	6	&	9	\\
%	Aglio		&	12	&	18	\\
%	Origano		&	2	&	3	\\
%\bottomrule
%\end{tabular}
%\end{table}

\end{method}

% Note
%\begin{note}
%\end{note}

% Figura
%\showit[1.25in]{example-image-b}{This is a picture}


