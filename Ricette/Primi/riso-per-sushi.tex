\recipe[Questa riso è perfetto per il sushi, ma può essere usato anche per una poke o per una classica insalata di riso. In commercio esiste il riso specifico per sushi, ma io preferisco usare il più economico e locale riso Roma.]{Riso per sushi, poke o insalata}
\serves{4}
\preptime{1 ora}
\cooktime[Cottura]{13 minuti}
\autore{Max}
%\vegetarian
%\freeze
\begin{ingreds}
	500g \index{riso} Roma
	Acqua
\columnbreak
\ingredients[Condimento per il riso sushi:]
	2 cucchiai di zucchero
	$\frac{1}{2}$ cucchiaio di sale
	1 bicchiere di aceto di riso
\end{ingreds}

\begin{method}
Lava il riso sotto l'acqua fredda mescolando delicatamente con la mano per togliere più amido possibile. Prosegui fino a che l'acqua non risulta limpida.

Metti il riso in un tegame e aggiungi acqua fredda fino ad arrivare a 2 dita sopra il livello del riso. Metti sul fuoco a fiamma vivace, quando inizia a bollire, metti il coperchio e abbassa il fuoco al minimo.

Occorreranno circa 13 minuti per arrivare alla cottura perfetta. Dopo 10 minuti controlla che non si stia asciugando troppo e il grado di cottura. Se si è asciugato troppo puoi aggiungere poca acqua calda.

Se vuoi preparare una insalata di riso o una poke puoi mettere il riso in una ciotola e condirlo come preferisci.

Se devi preparare il sushi questo è il momento di aggiungere il condimento che avrai prima scaldato al microonde per sciogliere perfettamente il sale e lo zucchero nell'aceto di riso.

\end{method}

%\showit[1.25in]{example-image-b}{This is a picture}


