\recipe[Negli anni 90 frequentavamo la Birreria Hobelix di Botteghe di Albinea (RE) che cucinava, e cucina ancora, questo primo particolarmente ricco e gustoso. Variando i pochi ingredienti sono riuscito ad ottenere una versione che mi soddisfa ed è diventata praticamente una ricetta tradizionale di famiglia che ti salva quando ci sono gruppi particolarmente numerosi.]{Penne all'ubriaca}
\serves{4}
\preptime{1 ora}
\cooktime[Cottura]{30 minuti}
\autore{Max}
%\freeze
\begin{ingreds}
	500g di penne
	500g di panna fresca
	2 salsicce
	2 fette di speck
	1 bottiglia di lambrusco secco e corposo
	Parmigiano Reggiano
%\columnbreak
%\ingredients[For the Crumble Mixture:]

\end{ingreds}

\begin{method}
Rosola bene le salsicce sgranate in una padella, aggiungi quasi a fine cottura lo speck tagliato a dadini poi fai sfumare con almeno mezza bottiglia di lambrusco. Lascia evaporare l'alcool e aggiungi la panna. Fai restringere il sugo a fuoco moderato.

Porta a cottura la pasta in acqua salata, poi fai mantecare nel sugo aggiungendo un mestolo di acqua di cottura.

Metti nei piatti e aggiungi abbondante Parmigiano Reggiano grattugiato.

\end{method}
\subsection*{Note}
	Accorgimenti importanti per replicare al meglio la ricetta: il vino deve essere secco, corposo e abbondante per dare un bel colore viola alla pasta; la panna deve essere fresca, meglio evitare la panna da cucina; il Parmigiano sopra alla pasta deve essere abbondantissimo.

	Questo è praticamente un piatto unico.
%\showit[1.25in]{example-image-b}{This is a picture}


