\recipe[]{Pan bauletto}
\serves{4}%<----Numero di porzioni
\preptime{3 ore}%<---Tempo di preparazione
\cooktime[]{40 minuti}%<-----Tempo di cottura
%\autore{QUI-IL_NOME}
\begin{ingreds}
	% Qui gli ingredienti uno sotto all'altro
	500g di farina
	200g di acqua
	80g di latte
	13g di lievito
	50g di olio di semi
	8g di sale
	un cucchiaino di malto o miele


% Se vuoi aggiungere gli ingredienti per una preparazione
% specifica della ricetta usa questi 2 comandi
%\columnbreak
%\ingredients[Per il ripieno:]

\end{ingreds}

\begin{method}
Sciogli il lievito nel latte appena tiepido con il miele. Impasta la farina con latte, miele e lievito, poi aggiungi poco alla volta l'acqua. Quando l'impasto è grezzo aggiungi il sale e l'olio. Impasta per 8 minuti. Fai riposare la pasta per 20 minuti. Poi stendila con il mattarello realizzando un rettangolo delle dimensioni dello stampo per il pan bauletto. Arrotola la pasta e fai la cucitura delle estremità. Metti a lievitare nello stampo imburrato coperto da un contenitore di plastica per mantenere l'umidità.
Cuoci in forno statico a \temp{190} per 40 minuti mettendo un tegamino di acqua calda nel forno.

% Esempio di tabella ---------->
%\begin{table}[h]
%\begin{tabular}{lcc}
%\toprule
%	Ingredienti	&	Peso(g)	&	Peso(g)\\
%\midrule
%	Paprika		&	40	&	60	\\
%	Sale		&	33	&	50	\\
%	Zucchero	&	20	&	30	\\
%	Cumino		&	2	&	3	\\
%	Pepe		&	6	&	9	\\
%	Aglio		&	12	&	18	\\
%	Origano		&	2	&	3	\\
%\bottomrule
%\end{tabular}
%\end{table}

\end{method}

%\subsection*{Note}

% Figura
%\showit[1.25in]{example-image-b}{This is a picture}


