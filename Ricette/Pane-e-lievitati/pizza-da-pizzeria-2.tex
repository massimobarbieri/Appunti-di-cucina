\recipe[Nuova ricetta della pizza 65\% di idratazione con puntata e appretto]{Pizza da pizzeria 2.0}
\serves{4}%<----Numero di porzioni
\preptime{1,30 ora per impasto compresi riposi}%<---Tempo di preparazione
\cooktime[]{10 minuti}%<-----Tempo di cottura
\autore{Max}
\begin{ingreds}
	600g di farina Caputo Pizzeria
	390g di acqua
	20g di sale
	1,5g di lievito secco Caputo
	Semola come farina di spolvero
	(ingredienti a piacere per la farcitura)

% Se vuoi aggiungere gli ingredienti per una preparazione
% specifica della ricetta usa questi 2 comandi
%\columnbreak
%\ingredients[Per il ripieno:]

\end{ingreds}

\begin{method}
\underline{Impasto da preparare il giorno prima}. Mettila farina nella planetaria e aggiungi l'acqua fredda di frigo fino a raggiungere il 55\% di idratazione. Aziona la planetaria a velocità bassa e impasta per 10/12 minuti, fino a che non si è legato l'impasto.

Aggiungi il lievito e prosegui aggiungendo l'acqua poco alla volta fino al 60\% di idratazione sempre azionando la planetaria a velocità bassa.

Aggiungi il sale e prosegui aggiungendo l'acqua restante poco alla volta. Per fare incordare puoi aumentare la velocità della planetaria.

Togli l'impasto dalla planetaria e dai un paio di pieghe. Fai dei riposi da 10 minuti fino a che l'impasto non risulta liscio.

Riponi l'impasto in un contenitore ermetico leggermente unto, fai riposare a temperatura ambiente per 1 ora (puntata) poi trasferisci in frigo per 16 ore circa.

Il giorno successivo togli l'impasto dal frigo (alle ore 14:30) prepara i panetti da 250g e lascia lievitare in un contenitore spolverato di semola per 4 ore abbondanti (appretto).

\underline{Cottura}. Preriscalda il forno a \temp{250}, poi accendi il grill alla massima temperatura, posizionando la griglia del forno nel ripiano più alto. Scalda una padella da piadina a fuoco medio. Tira la pizza usando la semola come farina di spolvero, mettila sulla padella calda e farciscila. Cuoci fino a che il fondo della pizza non è perfetto. A questo punto trasferisci la pizza sulla retina e cuoci in forno fino a che non prende un colore perfetto nella parte superiore. Ci vorranno circa 4 minuti.

\underline{Note per la farcitura}. Taglia la mozzarella a strisce di circa un centimetro di spessore e mettila scolare per togliere il latticello interno.


% Esempio di tabella ---------->
\begin{table}[h]
\centering
\begin{tabular}{lccccc}
\toprule
	Ingredienti			&	2 pizze	&	3 pizze	&	4 pizze	&	6 Pizze\\
\midrule
	Farina				&	300		&	450		&	600		&	900\\
	Acqua (65\% idra)	&	195(165+15+15)		&	292(248+22+22)		&	390(360+30+30)		&	585(495+45+45)\\
	Sale					&	10		&	15		&	20		&	30\\
	Lievito				&	0,7		&	1,1		&	1,5		&	2,2\\
\bottomrule
\end{tabular}
\end{table}

\end{method}

%\subsection*{Note}

% Figura
%\showit[1.25in]{example-image-b}{This is a picture}


