\recipe[]{Pizza in teglia}
\serves{4}
\preptime{1 ora}
\cooktime[]{30 minuti}
\autore{Max}
%\vegetarian
%\freeze
\begin{ingreds}
	1kg \index{farina} 1 + manitoba
	40g olio e.v.o.
	20 sale
	10g lievito di birra secco
	800g acqua
%\columnbreak
%\ingredients[For the Crumble Mixture:]
\end{ingreds}

\begin{method}
\underline{Impasto a mano.} Il pomeriggio del giorno prima metti in una ciotola la farina e il lievito poi mescola. Aggiungi poco alla volta l'acqua fino a incorporarla tutta. Poi aggiungi il sale, l'olio e mescola con un cucchiaio. L'impasto rimarrà grumoso. Aspetta 15 minuti poi metti l'impasto sulla spianatoia infarinata e piega in 3, prima da una parte poi dall'altra. Ripeti queste pieghe 3 o 4 volte lasciando un tempo di riposo di 15 minuti fra una piega e la successiva.

\underline{Impasto con la planetaria} Se vuoi usare la planetaria, metti tutti gli ingredienti nella ciotola e impasta per 8 minuti a velocità medio alta con la foglia. Poi rimuovi la pasta dalla foglia e fai riposare per 10 minuti. Fai partire la planetaria alla velocità minima giusto il tempo di raccogliere la pasta poi fai altri 10 minuti di riposo e ripeti l'operazione. Trasferisci l'impasto sul piano in acciaio oliato e fai le pieghe, lasciando riposare la pasta 10 minuti fra una piega e la successiva.

Metti l'impasto già diviso in 2 in un contenitore ermetico unto e fai lievitare in frigo per tutta la notte.

Il giorno successivo togli l'impasto dal frigo alle 15:00 e lascia a temperatura ambiente. Dopo un paio d'ore stendi sulla leccarda oliata e fai lievitare per 2 o 3 ore. Cuoci in forno statico a \temp{250} mettendo la leccarda nel ripiano più basso per 10 minuti. Poi abbassa a \temp{230} per 10 minuti al centro del forno.




%\begin{table}[h]
%\begin{tabular}{lcc}
%\toprule
%	Ingredienti	&	Peso(g)	&	Peso(g)\\
%\midrule
%	Paprika		&	40	&	60	\\
%	Sale		&	33	&	50	\\
%	Zucchero	&	20	&	30	\\
%	Cumino		&	2	&	3	\\
%	Pepe		&	6	&	9	\\
%	Aglio		&	12	&	18	\\
%	Origano		&	2	&	3	\\
%\bottomrule
%\end{tabular}
%\end{table}

\end{method}

%\showit[1.25in]{example-image-b}{This is a picture}


