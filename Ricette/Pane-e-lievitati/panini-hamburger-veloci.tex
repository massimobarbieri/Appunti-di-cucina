\recipe[]{Panini da hamburgher veloci}\label{panini-hamburger}
\serves{8}
\preptime{2 ore e 30}
\cooktime[Cottura]{30 minuti}
\autore{Max}
\begin{ingreds}
	500g di farina 0
	250g di acqua fredda
	30g di strutto
	10g di sale
	25g di lievito
\end{ingreds}

\begin{method}
	Metti tutti gli ingredienti nella planetaria e impasta per 10 minuti.

	Dai un paio di pieghe alla pasta e forma le palline. Per mini hamburgher fai palline da 30g per hamburgher tradizionali fai palline da 120g. Questa operazione si chiama pirlatura. Si tratta di richiudere la pasta su se stessa come un sacchetto in modo che la gabbia glutinica intrappoli le bolle d'aria che si sviluppano durante la lievitazione.

	Fai lievitare fino al raddoppio del volume. Servirà circa 1 ora e 30 minuti, ma questo tempo può variare in base alla temperatura e alla quantità di lievito che decidi di utilizzare.
	
	Quando i panini sono lievitati, se vuoi, puoi spennellare la superfice con tuorlo d'uovo e acqua o latte e successivamente cospargere semi si sesamo o altri semi. Questa operazione, oltre a rendere i panini più belli, conferirà ai panini una nota profumata interessante.

	Cuoci i panini in forno ventilato a \temp{190} per 15 minuti.

\end{method}
