\recipe[Dopo svariate prove questa è la versione che preferisco della pizza tonda tipo pizzeria fatta con il forno di casa. Permette di fare una pizza praticamente perfetta senza eccessive complicazioni]{Pizza da pizzeria}
\serves{4}%<----Numero di porzioni
\preptime{1 ora per impasto compresi riposi}%<---Tempo di preparazione
\cooktime[]{10 minuti}%<-----Tempo di cottura
\autore{Max}
\begin{ingreds}
	600g di farina 0 (Valeriani)
	360g di acqua
	20g di sale
	20g di olio e.v.o.
	10g di zucchero
	3g di lievito fresco
	Semola come farina di spolvero
	(ingredienti a piacere per la farcitura)

% Se vuoi aggiungere gli ingredienti per una preparazione
% specifica della ricetta usa questi 2 comandi
%\columnbreak
%\ingredients[Per il ripieno:]

\end{ingreds}

\begin{method}
\underline{Impasto da preparare il giorno prima}. Mescola tutta l'acqua con il lievito e metà della farina. Aggiungi lo zucchero poi fai riposare per 30 minuti. Aggiungi la farina poco alla volta e continua ad impastare. Poi aggiungi sale e olio ed impasta per almeno 7 minuti. Trasferisci l'impasto sulla spianatoia e fai alcune pieghe. Lascia riposare per 10 minuti poi fai altre pieghe. Dividi l'impasto in panetti da 250g e forma delle palline che metterai in un contenitore ermetico leggermente unto a lievitare in frigo per 1 giorno. Il giorno successivo togli i contenitori dal frigo il pomeriggio.

\underline{Cottura}. Preriscalda il forno a \temp{250}, poi accendi il grill alla massima temperatura, posizionando la griglia del forno nel ripiano più alto. Scalda una padella da piadina a fuoco medio. Tira la pizza usando la semola come farina di spolvero, mettila sulla padella calda e farciscila. Cuoci fino a che il fondo della pizza non è perfetto. A questo punto trasferisci la pizza sulla retina e cuoci in forno fino a che non prende un colore perfetto nella parte superiore. Ci vorranno circa 4 minuti.

\underline{Note per la farcitura}. Taglia la mozzarella a strisce di circa un centimetro di spessore e mettila scolare per togliere il latticello interno.


% Esempio di tabella ---------->
\begin{table}[h]
\centering
\begin{tabular}{lccccc}
\toprule
	Ingredienti	&	2 pizze	&	3 pizze	&	4 pizze	&	5 Pizze	&	6 Pizze\\
\midrule
	Farina		&	300		&	450		&	600		&	750		&	900\\
	Acqua		&	180		&	270		&	360		&	450		&	540\\
	Sale			&	10		&	15		&	20		&	25		&	30\\
	Olio			&	10		&	15		&	20		&	25		&	30\\
	Lievito		&	1,5		&	2,2		&	3		&	3,7		&	4,5\\
	Zucchero		&	5		&	7		&	10		&	12		&	14\\
\bottomrule
\end{tabular}
\end{table}

\end{method}

%\subsection*{Note}

% Figura
%\showit[1.25in]{example-image-b}{This is a picture}


