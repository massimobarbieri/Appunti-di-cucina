\recipe[Panini semplici con impasto diretto.]{Mantovane (Spaccatelle)}
\serves{13}%<----Numero di porzioni
\preptime{4 ore}%<---Tempo di preparazione
\cooktime[]{35 minuti}%<-----Tempo di cottura
\autore{Max}
\begin{ingreds}
	% Qui gli ingredienti uno sotto all'altro
	1Kg di farina 0
	450g di acqua
	10g di lievito fresco
	20g di sale
	12g di malto d'orzo
	
% Se vuoi aggiungere gli ingredienti per una preparazione
% specifica della ricetta usa questi 2 comandi
%\columnbreak
%\ingredients[Per il ripieno:]

\end{ingreds}

\begin{method}
Metti tutti gli ingredienti nella planetaria ad esclusione del sale (la planetaria impasta al massimo 750g di farina: vedi tabella con dosi differenti). Impasta leggermente poi aggiungi la farina e continua ad impastare a velocità elevata per 10 minuti.

Fai riposare l'impasto per venti minuti poi dividi in panetti da 120g. Crea un salsicciotto, come per fare gli gnocchi, appiattisci con il mattarello e arrotola per formare la mantovana. Fai lievitare in forno per 3 ore. Cuoci in forno ventilato a \temp{180} per 35 minuti mettendo la teglia di alluminio direttamente sopra le leccarde calde.

% Esempio di tabella ---------->
\begin{table}[h]
\begin{tabular}{lcc}
\toprule
	Ingredienti	&	Peso(g)	&	Peso(g)\\
\midrule
	Farina		&	1000		&	750	\\
	Acqua		&	450		&	338	\\
	Lievito		&	10		&	7,5	\\
	Olio			&	75		&	56	\\
	Sale			&	20		&	15	\\
	Malto		&	12		&	9	\\
\bottomrule
\end{tabular}
\end{table}

\end{method}

%\subsection*{Note}

% Figura
%\showit[1.25in]{example-image-b}{This is a picture}


