\recipe[]{Involtini di carne alla siciliana}
\serves{4}
\preptime{1 ora}
\cooktime[Cottura]{15 minuti}
\autore{Max}
\begin{ingreds}
	6 fette di carpaccio di manzo  \index{carpaccio di manzo}
	una cipolla rossa \index{cipolla}
	scamorza \index{scamorza}
	prosciutto cotto \index{prosciutto cotto} 
	uvetta \index{uvetta}
	pane raffermo \index{pane raffermo}
	pangrattato \index{pangrattato}
	un bicchiere di latte \index{latte}
	12 foglie di alloro \index{alloro}
	sale
	pepe
\end{ingreds}

\begin{method}
Taglia il carpaccio in modo da formare delle fette triangolari. In genere per ogni fetta di carpaccio è possibile ricavare 2 triangoli.

Prepara la farcia per gli involtini mescolando l'uvetta ammollata e leggermente tritata con il prosciutto cotto tagliato a cubettini, la scamorza e un po' di pane ammollato nel latte e strizzato. Regola di sale e pepe. Non ho specificato le quantità perché si può adattare al proprio gusto personale. Io in genere metto scamorza, prosciutto cotto e pane in parti uguali e sto un po' più scarso con l'uvetta.

Prepara gli involtini con la farcia; mettili in uno spiedo alternati da un petalo di cipolla e una foglia di alloro.

Ungi leggermente gli  spiedini e dai una spolverata con il pangrattato, poi cuoci al barbecue fino allo scioglimento completo della farcia.

\end {method}


