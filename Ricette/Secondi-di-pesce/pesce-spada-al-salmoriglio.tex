\recipe[Il salmoriglio è una salsa delicata, di origini siciliane, a base di olio extravergine di oliva limone e origano. In questa ricetta viene preparata a bagnomaria per estrarre al meglio gli olii essenziali profumati delle erbe aromatiche.]{Pesce spada al salmoriglio}
\serves{4}
\preptime{1 ora}
\cooktime[Cottura]{20 minuti}
\autore{Max}
\begin{ingreds}
	\ingredients[Per il salmoriglio:]
	1 limone con buccia edibile
	1/2 bicchiere di olio e.v.o.
	2 foglie di alloro
	1 spicchio d'aglio
	origano
	pepe
	sale
\columnbreak
     	600g Pesce spada
     	Olio e.v.o.
	Sale
\end{ingreds}

\begin{method}
Taglia il pesce spada a tocchetti di 3-4 centimetri, ungilo accuratamente e riponilo in frigorifero.

Inizia a preparare il salmoriglio aggiungendo ad una bull adatta al bagnomaria i seguenti ingredienti: la scorza di limone grattigiata, il succo di un limone, un pizzico di sale, un po' di pepe e un mestolo di acqua calda. Aggiungi l'aglio tritato, l'origano essiccato e le due foglie di alloro spezzettate. Mescola con una frusta per fare sciogliere bene il sale. Aggiungi l'olio extravergine di oliva e cuoci a bagnomaria per circa 15 minuti mescolando di tanto in tanto con la frusta. A cottura ultimata, togli la bull dal fuoco e filtra la salsa con un colino.

Cuoci in una padella antiaderente a fuoco alto il pesce spada, lasciandolo leggermente crudo all'interno. Occorreranno pochi minuti per lato. Lascia riposare il pesce cotto per alcuni minuti coprendolo con carta di alluminio in modo che il calore si diffonda all'interno e si ridistribuiscano i succhi. Taglia i pezzi di pesce in due e disponilo in un piatto poi bagna il pesce con il salmoriglio.

\end {method}



