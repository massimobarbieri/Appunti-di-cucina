\recipe[Il baccalà mantecato è una classica ricetta Veneta che ho riadattato abbassando il quantitativo di olio per renderla un po' più sana e meno calorica.]{Baccalà mantecato}
\serves{4}
\preptime{1 ora}
\cooktime[Cottura]{20 minuti}
\autore{Max}
\begin{ingreds}
	1 filetto di baccalà\index{baccalà}
	500g latte\index{latte} intero
	3 spicchi d'aglio\index{aglio}
	1 gambo di sedano\index{sedano}
	1 carota\index{carota}
	mezzo bicchiere di olio e.v.o.

%\columnbreak
%\ingredients[For the Crumble Mixture:]

\end{ingreds}

\begin{method}
Ammolla il baccalà nell'acqua fredda per 2 o 3 giorni cambiando l'acqua ogni 6 ore. Assaggialo ogni tanto per capire il giusto grado di salatura.

Metti il baccalà in un tegame e copri con latte e acqua, aggiungi le verdure e l'aglio e cuoci per 20 minuti.

Manteca nella planetaria con la frusta aggiungendo l'olio a filo, acqua di cottura. Per ottenere un risultato cremoso dovrai fare andare la planetaria per almeno 10 minuti. Io metto anche la pelle del baccalà che aiuta a legare la crema anche se lascia qualche puntino nero nella crema.


%\begin{table}[h]
%\begin{tabular}{lcc}
%\toprule
%	Ingredienti	&	Peso(g)	&	Peso(g)\\
%\midrule
%	Paprika		&	40	&	60	\\
%	Sale		&	33	&	50	\\
%	Zucchero	&	20	&	30	\\
%	Cumino		&	2	&	3	\\
%	Pepe		&	6	&	9	\\
%	Aglio		&	12	&	18	\\
%	Origano		&	2	&	3	\\
%\bottomrule
%\end{tabular}
%\end{table}

\end{method}

%\showit[1.25in]{example-image-b}{This is a picture}


