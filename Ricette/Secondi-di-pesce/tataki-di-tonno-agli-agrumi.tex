\recipe[Il tataki è una cottura particolare che prevede di scottare gli alimenti a temperatura altissima per poi immergerli in acqua e ghiaccio per bloccare la cottura.]{Tataki di tonno fagioli e cipolla}
\serves{4}
\preptime{1 ora}
\cooktime[Cottura]{5 minuti}
\begin{ingreds}
	500g di tonno fresco
	1 arancia
	1 rametto di rosmarino
	1 spicchio d'aglio
	olio evo
	sale
\columnbreak
\ingredients[Per la salsa:]
	120g di fagioli canellini lessati
	brodo vegetale
	limone
	olio evo
	sale
	cipolla
\end{ingreds}

\begin{method}

Taglia il tonno a parallelepipedi con il lato piccolo di 3x3 centimetri. Metti i filetti a marinare per qualche ora in frigorifero con le scorze degli agrumi e l'olio extravergine di oliva.

Scalda una padella antiaderente o di ferro a temperatura altissima e cuoci il tonno per pochi secondi per ogni lato. Successivamente immergi il tonno in acqua e ghiaccio per fermare la cottura. Asciuga il tonno e mettilo a marinare con rosmarino e aglio coprendolo compeltamente di olio extravergine di oliva.

Prepara la salsa frullando i fagioli con brodo vegetale, succo di limone, sale e olio extravergine di oliva. Setaccia la salsa così ottenuta con un colino.

Friggi la cipolla leggermente infarinata.

Componi il piatto con la salsa di fagioli sul fondo, dei cubetti di tonno, la cipolla fritta, qualche granello di sale e un giro di olio evo.
\end {method}
