\recipe[Il tataki è una cottura particolare che prevede di scottare gli alimenti a temperatura altissima per poi immergerli in acqua e ghiaccio.]{Tataki di tonno fagioli e cipolla}
\serves{4}
\preptime{1 ora}
\cooktime[Cottura]{5 minuti}
\begin{ingreds}
	500g di tonno fresco
	1 arancia
	1 rametto di rosmarino
	1 spicchio d'aglio
	olio evo
	sale
\columnbreak
\ingredients[Per la salsa:]
	120g di fagioli canellini lessati
	brodo vegetale
	limone
	olio evo
	sale
	cipolla
\end{ingreds}

\begin{method}

	Taglia il tonno a parallelepipedi e mettiola marinare per qualche ora in frigorifero con le scorze degli agrumi e l'olio extravergine di oliva.

	Scalda una padella a temperatura altissima e cuoci il tonno per pochi secondi per ogni lato. Poi immergi il tonno in acqua e ghiaccio per fermare la temperatura. Asciuga il tonno e mettilo a marinare con rosmarino e aglio coprendolo compeltamente di olio extravergine di oliva.

	Frulla i fagioli con brodo vegetale, succo di limone, sale e olio extravergine di oliva.

	Friggi la cipolla o disidratala in forno.

	Componi il piatto con la salsa di fagioli sul fondo, dei cubetti di tonno, la cipolla fritta, qualche granello di sale e un giro di olio evo.
\end {method}
