\recipe[Questa è la ricetta di mia mamma dei tortellini al forno. La particolarità di questa ricetta è il tempo di riposo del ripieno di circa 15gg. Il risultato è strepitoso.]{Tortellini al forno}
\serves{4}%<----Numero di porzioni
\preptime{1 ora}%<---Tempo di preparazione
\cooktime[]{30 minuti}%<-----Tempo di cottura
\autore{Max (Nonna Giuli)}
\begin{ingreds}
	% Qui gli ingredienti uno sotto all'altro
	\ingredients[Pasta frolla:]
	1kg farina
	400g burro
	2 tuorli
	2 uova
	2 bustine di vanillina
	100g di mandorle
	\frac{1}{2} bicchiere di liquore
\columnbreak
	\ingredients[Per il ripieno:]
	1,5Kg (secondo me è sbagliato) di marmellate miste \index{marmellata} \footnote{Devono essere tutte marmellate abbastanza acide: perfette prugne e ciliege, indispensabile il "Sapore" o Savuret}
	1 cucchiaio di polvere di caffé \index{caffé}
	2 cucchiai di cacao amaro \index{cacao}
	220g castagne bollite \index{castagne}
	70g uvetta ammollata \index{uvetta}
	Amaretti \index{amaretti}
	Mandorle \index{mandorle}


% Se vuoi aggiungere gli ingredienti per una preparazione
% specifica della ricetta usa questi 2 comandi
%\columnbreak
%\ingredients[Per il ripieno:]

\end{ingreds}

\begin{method}
Prepara il ripieno dei tortellini amalgamando tutti gli ingredienti. Poiché non tutte le marmellate sono uguali, le dosi sono indicative e sarà necessario assaggiare il composto per capire come correggerlo. Fai riposare il ripieno fuori al fresco per 10--15 giorni.

% Esempio di tabella ---------->
%\begin{table}[h]
%\begin{tabular}{lcc}
%\toprule
%	Ingredienti	&	Peso(g)	&	Peso(g)\\
%\midrule
%	Paprika		&	40	&	60	\\
%	Sale		&	33	&	50	\\
%	Zucchero	&	20	&	30	\\
%	Cumino		&	2	&	3	\\
%	Pepe		&	6	&	9	\\
%	Aglio		&	12	&	18	\\
%	Origano		&	2	&	3	\\
%\bottomrule
%\end{tabular}
%\end{table}

\end{method}

%\subsection*{Note}

% Figura
%\showit[1.25in]{example-image-b}{This is a picture}


