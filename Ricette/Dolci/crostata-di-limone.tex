\recipe[Questa è la mia torta preferita che prepara molto spesso mia mamma. Non so da dove l'abbia presa. La crosticina leggermente croccante sopra, il profumo di mandorle e limone secondo me la rendono fantastica.]{Crostata di limone e mandorle}
\serves{8}
\preptime{1 ora}
\cooktime[Cottura]{40 minuti}
\autore{Max (Nonna Giuli)}
%\vegetarian
%\freeze
\begin{ingreds}
	200g mandorle tritate (o farina di mandorle)
	200g zucchero
	3 uova
	pasta frolla (realizzata con un uovo)
	buccia di 2 limone
	marmellata
	cacao amaro
	un bicchiere di sassolino

%\columnbreak
%\ingredients[For the Crumble Mixture:]

\end{ingreds}

\begin{method}
Sbollenta per pochi minuti le bucce dei limoni poi frulla con lo zucchero. Trita le mandorle e aggiungile al composto di limoni e zucchero e il tuorlo d'uovo. Se ti piace aggiungi anche un bicchiere di sassolino o altro liquore aromatico. Montagli albumi e incorporali al composto.

Stendi la pasta frolla su una teglia imburrata e infarinata. Metti sul fondo un leggero strato di marmellata, dai una spolverata di cacao amaro e versa il composto preparato in precedenza.

Cuoci a forno ventilato a \temp{160} per 40 minuti.

%\begin{table}[h]
%\begin{tabular}{lcc}
%\toprule
%	Ingredienti	&	Peso(g)	&	Peso(g)\\
%\midrule
%	Paprika		&	40	&	60	\\
%	Sale		&	33	&	50	\\
%	Zucchero	&	20	&	30	\\
%	Cumino		&	2	&	3	\\
%	Pepe		&	6	&	9	\\
%	Aglio		&	12	&	18	\\
%	Origano		&	2	&	3	\\
%\bottomrule
%\end{tabular}
%\end{table}

\end{method}

%\showit[1.25in]{example-image-b}{This is a picture}


