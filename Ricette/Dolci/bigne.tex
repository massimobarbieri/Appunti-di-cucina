\recipe[]{Bigné (Pasta choux)}
\serves{4}
\preptime{1 ora}
\cooktime[Cottura]{25 minuti}
\autore{Max}
%\vegetarian
%\freeze
\begin{ingreds}
	4 uova (210g) \index{uova}
	130g farina\index{farina} 00
	100g burro \index{burro}
	5g zucchero semolato
	200g acqua
	un pizzico di sale

%\columnbreak
%\ingredients[For the Crumble Mixture:]

\end{ingreds}

\begin{method}
Metti sul fuoco in un tegame di acciaio il burro e l'acqua; porta a bollore e aggiungi la farina setacciata. Mescola fino a che sul fondo del tegame non si forma una patina bianca.
Fai raffreddare la pasta nella planetaria utilizzando la foglia. Una volta raffreddata incorpora un uovo alla volta.
Metti l'impasto nella sacca a poche e, su una teglia ricoperta di carta da forno, prepara i bigné mettendo l'impasto delle dimensioni di una noce.

	Cuoci in forno statico a \temp{220} per 15 minuti, poi abbassa a \temp{190} per 10 minuti. Poi fai raffreddare con lo sportello leggermente aperto.
In alternativa puoi cuocere a forno ventilato a \temp{175} per 30 minuti, facendo raffreddare con lo sportello del forno leggermente aperto.

	Una volta raffrettati, i bigné possono essere farciti a piacere. Il più classico ripieno è la crema pasticcera (vedi ricetta \ref{crema-pasticcera}).
\end{method}

\subsection*{Note}
	La parte cruciale di questa ricetta è la cottura della pasta choux in padella. Se cuoci troppo poco poi avranno troppa umidità nel forno e i bigné non lieviteranno correttamente.

%\showit[1.25in]{example-image-b}{This is a picture}


