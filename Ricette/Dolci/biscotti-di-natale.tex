\recipe[Questa è una ricetta di famiglia che realizziamo una volta all'anno nel periodo delle feste Natalizie. Prima delle festività ne prepariamo svariati chili da regalare ad amici e parenti. Si conservano per almeno un mese in ambiente fresco e asciutto.]{Biscotti di Natale}
\serves{4}
\preptime{2 ora}
\cooktime{25 minuti}
\autore{Max}
\begin{ingreds}
	500g farina tipo 1
	230g noci e nocciole
	200g cioccolato
	200g burro (puoi abbassare 180g)
	200g zucchero (puoi abbassare a 180g)
	3 uova
	1 bustina di lievito
	un pizzico di sale
	aromi (vanillina, vaniglia, limone)

\end{ingreds}

\begin{method}
Puoi realizzare l'impasto per questi biscotti anche con la planetaria utilizzando lo strumento a forma di foglia.

Mescola le uova con lo zucchero, poi aggiungi il burro a temperatura ambiente e continua a mescolare. Aggiungi qualche goccia di olio essenziale di limone o vaniglia. Aggiungi la farina, il lievito.

Quando l'impasto risulta omogeneo aggiungi le noci, le nocciole e, per ultimo, il cioccolato: distribuisci questi ultimi ingredienti senza impastare troppo per non sciogliere il cioccolato.
	
Prepara dei salsicciotti di circa 4cm di diametro aiutandoti con un pezzo di carta da forno che utilizzerai per avvolgere il salsicciotto stesso e riporlo successivamente nel congelatore. I biscotti dovranno rimanere nel congelatore per un minimo di 30 minuti, oppure puoi congelarli completamente e cuocerli in un secondo tempo.
	
Taglia biscotti e cuoci a \temp{185} per 20/25 minuti.

Fai raffreddare.

\end{method}

\subsection*{Note}
		Puoi sostituire lo zucchero con 20g di stevia ma dovrai aggiungere circa mezzo bicchiere di latte e cuocere per circa 10 minuti in più.
		
		Con 500g di farina puoi produrre 4 salsicciotti di biscotti.
