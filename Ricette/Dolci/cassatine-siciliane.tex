\recipe[]{Cassatine siciliane}
\serves{6}
\preptime{2 ore}
\cooktime[]{8 minuti}
%\vegetarian
%\freeze
\begin{ingreds}
\ingredients[Pasta di mandorle:]
	150g farina di mandorle
	75g sciroppo di glucosio
	50g zucchero a velo
	5-10g acqua
	colorante verde
\columnbreak
\ingredients[Pan di Spagna:]
	2 uova
	60g farina 00
	40g zucchero
	20g maizena
\ingredients[Crema di ricotta:]
	400g ricotta di pecora
	120g zucchero
	32g cioccolato

\end{ingreds}

\begin{method}
\underline{Pasta di mandorle.} Impasta la farina di mandorle con la fogli aggiungendo gli zuccheri e per ultima l'acqua. Fai un salsicciotto e cospargilo di zucchero a velo, poi avvolgilo nella pellicola trasparente e fai riposare a temperatura ambiente.

\underline{Pan di Spagna}. Con la frusta monta le uova per 8 minuti a velocità medio alta aggiungendo lo zucchero in 2 o 3 volte. Incorpora le farine al composto poco alla volta. Stendi su una teglia ricoperta di carta da forno. Inforna in forno ventilato a \temp{190} per 6-8 minuti. Cospargi la superficie con zucchero a velo e ribalta su un foglio di carta da forno.

\underline{Crema di ricotta}. Setaccia la ricotta, incorpora lo zucchero e aggiungi le gocce o i pezzetti di cioccolato. Puoi aggiungere anche, se ti piacciono, le arance candite.

\underline{Assemblaggio}. Stendi la pasta di mandorle alta 3 mm. Tagliala con un coppa pasta grande la pasta di mandorle. Cospargi di zucchero a velo lo stampino per le cassate e stendi la pasta. Taglia con un coppa pasta piccolo la cupola della cassata nella quale andrai a posizionare un disco di pan di Spagna inumidito delle stesse dimensioni. Riempi lo stampo di crema di ricotta e chiudi il fondo con un disco di pan di Spagna inumidito.


% Aggiungi disegno della cassatina assemblata


%\begin{table}[h]
%\begin{tabular}{lcc}
%\toprule
%	Ingredienti	&	Peso(g)	&	Peso(g)\\
%\midrule
%	Paprika		&	40	&	60	\\
%	Sale		&	33	&	50	\\
%	Zucchero	&	20	&	30	\\
%	Cumino		&	2	&	3	\\
%	Pepe		&	6	&	9	\\
%	Aglio		&	12	&	18	\\
%	Origano		&	2	&	3	\\
%\bottomrule
%\end{tabular}
%\end{table}

\end{method}

\showit[2.25in]{img/cassatina}{Assemblaggio della cassatina}


