\recipe[Preparare la nocciolata in casa è molto facile. Il risultato è migliore dei prodotti disponibili in commercio anche se il costo di preparazione è superiore, soprattutto se si prendono nocciole di qualità.]{Nocciolata}
\serves{-}%<----Numero di porzioni
\preptime{30 minuti}%<---Tempo di preparazione
\cooktime[]{10 minuti}%<-----Tempo di cottura
\autore{Max}
\begin{ingreds}
	% Qui gli ingredienti uno sotto all'altro
	500g di nocciole
	1 cucchiaio raso di cacao amaro
	50g di cioccolato (quello che preferisci)
	2 cucchiai di olio e.v.o.

% Se vuoi aggiungere gli ingredienti per una preparazione
% specifica della ricetta usa questi 2 comandi
%\columnbreak
%\ingredients[Per il ripieno:]

\end{ingreds}

\begin{method}
	Per preparare la nocciolata in casa è necessario un frullatore molto potente (power blender). Tosta le nocciole in forno o in padella per 10 minuti. Versa le nocciole che avrai fatto raffreddare nel frullatore e frulla prima a intermittenza poi in modo continuo aumentando gradualmente la velocità. Usa una spatola per spingere giù la pasta che si formerà. Aggiungi l'olio e.v.o. e continua a frullare fino ad ottenere il burro di nocciole. Fai sciogliere a bagnomaia il cioccolato tritato e aggiungi al burro di nocciole. Io preferisco usare il cioccolato fondente.

	La crema si conserva in frigorifero. Non so quanto si conserva, perché a casa mia dura sempre molto poco.


\end{method}

% Note
%\begin{note}
%\end{note}

% Figura
%\showit[1.25in]{example-image-b}{This is a picture}


