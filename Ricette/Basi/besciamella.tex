\recipe[]{Besciamella}\label{besciamella}
\serves{-}%<----Numero di porzioni
\preptime{15 minuti}%<---Tempo di preparazione
\cooktime[]{10 minuti}%<-----Tempo di cottura
\autore{Max}
\begin{ingreds}
	% Qui gli ingredienti uno sotto all'altro
	500g latte intero\index{latte}
	50g burro (anche meno)\index{burro}
	50g farina 00\index{farina}
	Noce moscata (o altri aromi)
	Sale

% Se vuoi aggiungere gli ingredienti per una preparazione
% specifica della ricetta usa questi 2 comandi
%\columnbreak
%\ingredients[Per il ripieno:]

\end{ingreds}

\begin{method}
Scalda il latte intero senza portarlo a bollore. Fai sciogliere il burro in una casseruola, aggiungi la farina e falla imbiondire mescolando con un cucchiaio di legno.

Aggiungi poco alla volta il latte, mescolando vigorosamente con una frusta per evitare la formazione di grumi.

Quando avrai versato tutto il latte riporta sul fuoco e cuoci per portare alla consistenza desiderata. Aggiungi la noce moscata o altri aromi e regola di sale.

\end{method}

\subsection*{Note}
La ricetta base della besciamella che si presta a diverse varianti (es. olio e.v.o. al posto del burro e brodo vegetale al posto del latte. Se dovessero formarsi dei grumi puoi frullare con un frullatore a immersione.

% Figura
%\showit[1.25in]{example-image-b}{This is a picture}


