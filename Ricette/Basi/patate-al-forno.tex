\recipe[]{Patate al forno}
\serves{4}%<----Numero di porzioni
\preptime{1 ora}%<---Tempo di preparazione
\cooktime[]{30 minuti}%<-----Tempo di cottura
%\autore{QUI-IL_NOME}
\begin{ingreds}
	% Qui gli ingredienti uno sotto all'altro
1Kg di patate
olio e.v.o.
rosmarino
4 spicchi d'aglio
sale
Altri aromi a piacere


% Se vuoi aggiungere gli ingredienti per una preparazione
% specifica della ricetta usa questi 2 comandi
%\columnbreak
%\ingredients[Per il ripieno:]

\end{ingreds}

\begin{method}
Lava le patate, togli eventuali parti brutte e tagliale a spicchi tenendo la buccia. Immergile in acqua fredda e cambia l'acqua più volte fino a che non risulta limpida, in modo che tutto l'amido sulla superficie della patata venga eliminato.

Metti sul fuoco una pentola di acqua, aggiungi una presa di sale e, quando raggiunge il bollore, tuffa metà delle patate. Quando riprende il bollore cuoci le patate per 3 minuti. Nel frattempo prepara una ciotola con abbondante olio evo e qualche ciuffetto di rosmarino.

Scola le patate con un mestolo forato e mettile nella ciotola con l'olio e il rosmarino. Mescola, poi chiudi la ciotola con un coperchio o con della pellicola trasparente. In questa fase puoi aggiungere altri aromi. Una variante interessante consiste nell'aggiungere un cucchiaino di curry.

Ripeti l'operazione con le patate restanti.

Riscalda il forno ventilato a \temp{230}. Disponi le patate sulla leccarda ricoperta di carta da forno avendo cura di non sovrapporle. Sala le patate e, se necessario, aggiungi ancora un po' di olio e.v.o.; riponi le patate in forno.

Quando le patate avranno la superficie leggermente bruciacchiata, impiegheranno circa 25 minuti, girale e aggiungi gli spicchi d'aglio in camicia. Aggiungendo gli spicchi solo a metà cottura, potrai profumare le patate e avere l'aglio cremoso che potrà essere spalmato sul pane o altro.

Completa la cottura delle patate fino a che non avranno anche l'altro lato della superficie bruciacchiata.

Il tempo di cottura totale delle patate è di circa un ora, ma il modo migliore per decidere quando toglierle dal forno è controllare il colore della superficie.

% Esempio di tabella ---------->
%\begin{table}[h]
%\begin{tabular}{lcc}
%\toprule
%	Ingredienti	&	Peso(g)	&	Peso(g)\\
%\midrule
%	Paprika		&	40	&	60	\\
%	Sale		&	33	&	50	\\
%	Zucchero	&	20	&	30	\\
%	Cumino		&	2	&	3	\\
%	Pepe		&	6	&	9	\\
%	Aglio		&	12	&	18	\\
%	Origano		&	2	&	3	\\
%\bottomrule
%\end{tabular}
%\end{table}

\end{method}

%\subsection*{Note}

% Figura
%\showit[1.25in]{example-image-b}{This is a picture}


