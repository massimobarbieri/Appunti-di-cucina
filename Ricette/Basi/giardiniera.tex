\recipe[Scrivo qui gli appunti relativi alle diverse prove che sto facendo con le verdure fermentate]{Giardiniera di verdure fermentate}
\serves{4}%<----Numero di porzioni
\preptime{10 giorni}%<---Tempo di preparazione
\cooktime[]{-}%<-----Tempo di cottura
\autore{Max}
\begin{ingreds}
	% Qui gli ingredienti uno sotto all'altro
	Verdure varie
	Salamoia con 20g di sale per litro d'acqua


% Se vuoi aggiungere gli ingredienti per una preparazione
% specifica della ricetta usa questi 2 comandi
%\columnbreak
%\ingredients[Per il ripieno:]

\end{ingreds}

\begin{method}
\underline{Giardiniera \#1}. Messo in salamoia (20g/1l) carote, cipolla di tropea, peperone rosso, cavolfiore viola. Lasciato in salamoia per 9 giorni. Risultato: colore bellissimo, la cipolla da un profumo ottimo ma si sfalda, peperone troppo molle, il resto delle verdure fantastico.

\underline{Giardiniera \#2}. 06/12/2022. Messo in salamoia (20g/1l) carote, cipolla di tropea, finocchio, cavolo romanesco, pepe. Risultato - Il colore non mi piace e il cavolo romanesco ha rilasciato un odore non sempre buono. Meglio abbondare con la cipolla.


% Esempio di tabella ---------->
%\begin{table}[h]
%\begin{tabular}{lcc}
%\toprule
%	Ingredienti	&	Peso(g)	&	Peso(g)\\
%\midrule
%	Paprika		&	40	&	60	\\
%	Sale		&	33	&	50	\\
%	Zucchero	&	20	&	30	\\
%	Cumino		&	2	&	3	\\
%	Pepe		&	6	&	9	\\
%	Aglio		&	12	&	18	\\
%	Origano		&	2	&	3	\\
%\bottomrule
%\end{tabular}
%\end{table}

\end{method}

%\subsection*{Note}

% Figura
%\showit[1.25in]{example-image-b}{This is a picture}


