\section{Temperature di cottura al cuore}
Conoscere la temperatura di cottura al cuore di un alimento è utile per poter replicare le ricette senza possibilità di errore e per garantire sempre il miglior risultato possibile secondo il proprio gusto personale.

La tabella \ref{temperature-al-cuore} è stata elaborata sulla base delle mie preferenze facendo prove con valori diversi.



\begin{table}
\begin{tabular}{llp{0.6\textwidth}}
\toprule
Alimento				&	Temp(°C)		&		Note		\\
\midrule
Hamburger			& 	62°C 		& Ideale per una cottura media\\
Arista di maiale	& 	71°C 		& Temperatura adatta per il maiale in genere che risulterà sicuro e succoso. Non è adatta per cotture di tagli duri tipo costine o pulled pork. \\
Pollo ripieno 		& 	71°C 		& Temperatura ideale per qualunque volatile ripieno\\
Polpettone di manzo	&	75°C 		& Temperatura ideale se si aggiunge pane ammollato nel latte, altrimenti meglio abbassare un po' la temperatura\\
\bottomrule
\end{tabular}
\label{temperature-al-cuore}
\caption{Temperature di cottura al cuore dell'alimento ottimali}
\end{table}

\section{Appunti di cottura a bassa temperatura (CBT)}
La cottura a bassa temperatura (CBT) è una tecnica che consiste nella cottura di un alimento imbustato sotto vuoto in bagno termostatico ad una temperatura che in genere varia dai \temp{52} agli \temp{80}.

I tempi di cottura con questa tecnica variano da un minimo di 30 minuti, ma possono arrivare anche alle 36-48 ore. In questa tecnica di cottura il tempo risulta spesso utile a sciogliere il collagene e a rendere teneri anche tagli di carne particolarmente duri, tuttavia, poiché il parametro delle temperatura è costate, spesso variare il tempo non comporta grandi cambiamenti nel risultato finale. Nella tabella \ref{temperature-cbt} ho sintetizzato i parametri di cottura che preferisco.

Se sei interessato a questa tecnica di cottura e stai leggendo questo testo, ti consiglio di leggere guide più esaustive soprattutto per chiarire alcuni aspetti legati alla sicurezza alimentare.

\begin{table}
\begin{tabular}{lllp{0.5\textwidth}}
\toprule
Alimento				&	Temp			&	Tempo	&	Note	\\
\midrule
Uovo					&	\temp{63}	&	1 ora	&	Simile all'uovo in camicia \\
Costata bovino		&	\temp{52}	&	2-3 ore	&	Per una cottura al sangue perfetta \\
Vitello tonnato		&	\temp{58}	&	4-5 ore &	\\
Costine di maiale	&	\temp{62}	&	12 ore	&	Poi rifinite in forno \\
Roast beef			&	\temp{58}	&	4-5 ore	&	Poi rifinito in padella in ferro \\
Filetti di sgombro	&	\temp{60}	&	40 minuti \\
\bottomrule
\end{tabular}
\label{temperature-cbt}
\caption{Parametri di cottura CBT.}
\end{table}

