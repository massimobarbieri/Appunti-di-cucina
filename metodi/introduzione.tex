\section{Cosa contiene questo testo}




\section{Temperature di cottura}
Conoscere la temperatura di cottura al cuore di un alimento è fondamentale per poter replicare le ricette senza possibilità di errore e per garantire sempre il miglior risultato possibile anche secondo il proprio gusto personale.

La tabella \ref{temperature-al-cuore} è stata elaborata sul mio gusto personale e facendo prove diverse.
La tabella
\begin{center}
\begin{tabular}{ll}
\toprule
Alcaloide & Origine \\
\midrule
atropina & belladonna \\
morfina
& papavero \\
nicotina & tabacco \\
\bottomrule
\end{tabular}
\end{center}
mostra l’origine di tre alcaloidi.



\begin{table}
\begin{tabular}{llp{0.6\textwidth}}
\toprule
Alimento				&	Temp(°C)		&		Note		\\
\midrule
Hamburger			& 	62°C 		& Ideale per una cottura media\\
Braciole di maiale	& 	72°C 		& Questa temperatura è adatta per la cottura del maiale in genere. Con questa temperatura il maiale risulterà sicuro e succoso.\\
Pollo ripieno 		& 	71°C 		& Temperatura ideale per qualunque volatile ripieno\\
Polpettone 			&	75°C 		& Temperatura ideale se si aggiunge pane ammollato nel latte, altrimenti meglio abbassare la temperatura\\
\bottomrule
\end{tabular}
\label{temperature-al-cuore}
\caption{Temperature di cottura al cuore dell'alimento ottimali}
\end{table}

